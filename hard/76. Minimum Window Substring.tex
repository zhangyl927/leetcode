给你一个字符串 S、一个字符串 T,请在字符串 S 里面找出:包含 T 所有字母的最小子串。

示例:

输入: S = "ADOBECODEBANC", T = "ABC"
输出: "BANC"
说明:

如果 S 中不存这样的子串,则返回空字符串 ""。
如果 S 中存在这样的子串,我们保证它是唯一的答案。
































思路:
创建两个大小为 256的 vector sArr 和 tArr,tArr 中存放 t 的字符出现次数。
遍历 s 的数组每次找到在 t 中出现的字符处,通过两个指针 left 和 right;

先使得 left 为第一个 s 中出现的 t 中的字符,若left = s.length()(即 s 中不含有 t 中的字符),return "";
令 right = left,right 向后移动,直到 [left, right] 中包含 t 中的所有字符,结果保存在 res 中。再移动 left,若此时仍然 [left, right] 中包含 t 中的所有字符,更新 res,否则,移动 right 指针。
当 right 指针移动到 right = s.length() 时,结束。

































code:
class Solution {
public:
    int findNextStrIdx(int start, string& s, vector<int>& tArr)
    {
        while(start < s.length())
        {
            if(tArr[s[start]] != 0) return start;
            else start++;
        }
        return start;
    }
    
    string minWindow(string s, string t) {
        string res;
        int matchCount = 0;
        vector<int> sArr(256, 0);
        vector<int> tArr(256, 0);
        for(auto c:t) tArr[c]++;
        
        int left = findNextStrIdx(0, s, tArr);
        if(left == s.length()) return ""; // s have no char in T
        
        int right = left; // find start && make right=left
        while(right < s.length())
        {
            int rightChar = s[right];
            if(sArr[rightChar] < tArr[rightChar]) matchCount++;
            sArr[rightChar]++;
            
            while(right<s.length() && matchCount==t.length())
            {
                if(res.empty() || res.length()>right-left+1) 
                    res=s.substr(left, right-left+1);
                int leftChar = s[left];
                if(sArr[leftChar] == tArr[leftChar]) matchCount--;
                sArr[leftChar]--;
                
                left = findNextStrIdx(left+1, s, tArr);
                
            }
            right = findNextStrIdx(right+1, s, tArr);
        }
        return res;
    }
};