给定一个数组,它的第 i 个元素是一支给定的股票在第 i 天的价格。

设计一个算法来计算你所能获取的最大利润。你最多可以完成 k 笔交易。

注意: 你不能同时参与多笔交易(你必须在再次购买前出售掉之前的股票)。

示例 1:

输入: [2,4,1], k = 2
输出: 2
解释: 在第 1 天 (股票价格 = 2) 的时候买入,在第 2 天 (股票价格 = 4) 的时候卖出,这笔交易所能获得利润 = 4-2 = 2 。
示例 2:

输入: [3,2,6,5,0,3], k = 2
输出: 7
解释: 在第 2 天 (股票价格 = 2) 的时候买入,在第 3 天 (股票价格 = 6) 的时候卖出, 这笔交易所能获得利润 = 6-2 = 4 。
     随后,在第 5 天 (股票价格 = 0) 的时候买入,在第 6 天 (股票价格 = 3) 的时候卖出, 这笔交易所能获得利润 = 3-0 = 3 。























思路:
股票问题框架解法。

不是太理解?
for(int j = k; j > 0; j--)
            {
                dp_k0[j] = max(dp_k0[j], dp_k1[j]+prices[i]);
                dp_k1[j] = max(dp_k1[j], dp_k0[j-1]-prices[i]);
            }





















code:
class Solution {
public:
    int maxProfit(vector<int>& prices) {
        int dp_i_0 = 0, dp_i_1 = INT_MIN;
        for(int i = 0; i < prices.size(); i++)
        {
            dp_i_0 = max(dp_i_0, dp_i_1+prices[i]);
            dp_i_1 = max(dp_i_1, dp_i_0-prices[i]);
        }
        return dp_i_0;
    }
    
    int maxProfit(int k, vector<int>& prices) {
        if(k > prices.size()/2)
            return maxProfit(prices);
        
        vector<int> dp_k0(k+1, 0);
        vector<int> dp_k1(k+1, INT_MIN);
        for(int i = 0; i < prices.size(); i++)
        {
            for(int j = k; j > 0; j--)
            {
                dp_k0[j] = max(dp_k0[j], dp_k1[j]+prices[i]);
                dp_k1[j] = max(dp_k1[j], dp_k0[j-1]-prices[i]);
            }
        }
        return dp_k0[k];
    }
};