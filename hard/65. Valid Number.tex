验证给定的字符串是否可以解释为十进制数字。

例如:

"0" => true
" 0.1 " => true
"abc" => false
"1 a" => false
"2e10" => true
" -90e3   " => true
" 1e" => false
"e3" => false
" 6e-1" => true
" 99e2.5 " => false
"53.5e93" => true
" --6 " => false
"-+3" => false
"95a54e53" => false

说明: 我们有意将问题陈述地比较模糊。在实现代码之前,你应当事先思考所有可能的情况。这里给出一份可能存在于有效十进制数字中的字符列表:

数字 0-9
指数 - "e"
正/负号 - "+"/"-"
小数点 - "."
当然,在输入中,这些字符的上下文也很重要。


























思路:
"	+532.345e-93	"
从前往后判断,第一个不为‘ ’的位为 +/-/isdigit,digits 之后是 ‘.’;
1. 判断 +/-;
2. 判断 '.' 前的数;
3. 判断 '.' 后的数;
4. 判断 e;
5. 判断 e 之后的 符号;
6. 判断数字;
7. 判断是否到结尾。






























code:
class Solution {
public:
    bool isNumber(string s) {
        if(s.length() == 0) return false;
        int i = 0, n = s.length();
        // check start blank space
        while(i<n && s[i]==' ') i++;
        // check char +/-
        if(i<n && (s[i]=='+' || s[i]=='-')) i++;
        // check digit until .
        bool isDigits = false;
        while(i<n && isdigit(s[i]))
        {
            i++; isDigits = true;
        }
        // check post . parts
        if(i<n && s[i]=='.')
        {
            i++;
            while(i<n && isdigit(s[i]))
            {
                isDigits = true; i++;
            }
        }
        if(i<n && s[i]=='e' && isDigits)
        {
            i++;
            isDigits = false;
            // check +/- sign.
            if(i<n && (s[i]=='+' || s[i]=='-')) i++;
            while(i<n && isdigit(s[i]))
            {
                i++; isDigits = true;
            }
        }
        // 
        while(i<n && s[i]==' ') i++;
        return isDigits && i == s.length();
    }
};