给定一个仅包含小写字母的字符串,去除字符串中重复的字母,使得每个字母只出现一次。需保证返回结果的字典序最小(要求不能打乱其他字符的相对位置)。

示例 1:

输入: "bcabc"
输出: "abc"
示例 2:

输入: "cbacdcbc"
输出: "acdb"

























思路一:
遍历字符串数组,将字符出现的次数保存在 cnt 中。used 中存放字符是否被使用;
string sb 存放结果。遍历字符串数组,当前字符为 c,
if c is used,continue;
else 
	while sb length()>0 && c<sb.back() && count[sb.back()]>0,sb.pop();
	sb.append(c); used[c] = true;


思路二:
先将 s 存放在数组长度为 26 的 sv 中,统计每个字符出现的次数。
遍历 s,for(auto c:s)
		if 返回字符串 res 为空时, 直接 s += c,sv[c-'a']--;
		else if c 在 res 中,直接将 sv[c-'a']--, continue;
		else 
			while ( 判断 c 与 res 末尾的字符 cm 比较,若 c < cm && sv[cm-'a'] > 0)
				res.pop_back();
			res += c;
		sv[c-'a']--;





















code2:
class Solution {
public:
    string removeDuplicateLetters(string s) {
        vector<int> sv(26);
        for(auto i:s) sv[i-'a']++;
        string res;
        for(auto c:s) {
            if(res.find(c) != string::npos) {
                sv[c-'a']--;
                continue;
            }
            if(res.empty()) res += c;
            else {
                while(c < res.back() && sv[res.back()-'a'] > 0) {
                    res.pop_back();
                }
                res += c;
            }
            sv[c-'a']--;
        }
        return res;
    }
};










code:
class Solution {
public:
    string removeDuplicateLetters(string s) {
        string sb;
        vector<int> count(26, 0);
        vector<bool> used(26, false);
        for(auto c:s) count[c-'a']++;
        for(int i = 0; i < s.length(); i++)
        {
            count[s[i]-'a']--;
            // char is already in sb
            if(used[s[i]-'a']) continue;
            
            // char not in sb and char<sb.back() and sb.back().count > 0
            while(sb.size()>0 && sb.back()>s[i] && count[sb.back()-'a']>0)
            {
                used[sb.back()-'a'] = false;
                sb.pop_back();
            }
            sb += s[i];
            used[s[i]-'a'] = true;
        }
        return sb;
    }
};