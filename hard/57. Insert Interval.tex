给出一个无重叠的 ,按照区间起始端点排序的区间列表。

在列表中插入一个新的区间,你需要确保列表中的区间仍然有序且不重叠(如果有必要的话,可以合并区间)。

示例 1:

输入: intervals = [[1,3],[6,9]], newInterval = [2,5]
输出: [[1,5],[6,9]]
示例 2:

输入: intervals = [[1,2],[3,5],[6,7],[8,10],[12,16]], newInterval = [4,8]
输出: [[1,2],[3,10],[12,16]]
解释: 这是因为新的区间 [4,8] 与 [3,5],[6,7],[8,10] 重叠。
































思路:
插入区间的起始: nstart, 结束:nend
遍历所有区间,若某个区间的 end < nstart,表明该区间是需要合并的第一个区间,若找不到,则将插入区间直接插入到原有区间序列的末尾。继续向后遍历找,若某个区间的 start <= nend,则该区间也需要合并,直到某个区间的 start > nend 为止;
































code:
class Solution {
public:
    vector<vector<int>> insert(vector<vector<int>>& intervals, vector<int>& newInterval) {
        vector<vector<int>> res;
        int i = 0, n = intervals.size();
        int nstart = newInterval[0];
        int nend = newInterval.back();
        while(i<n && intervals[i][1]<nstart)
        {
            res.push_back(intervals[i]); i++;
        }
        if(i == n)
        {
            res.push_back(newInterval);
            return res;
        }
        nstart = min(intervals[i][0], nstart);
        while(i<n && intervals[i][0]<=nend)
        {
            nend = max(intervals[i][1], nend);
            i++;
        }
        res.push_back({nstart, nend});
        while(i < n) res.push_back(intervals[i++]);
        return res;
    }
};