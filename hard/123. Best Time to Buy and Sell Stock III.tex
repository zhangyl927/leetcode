给定一个数组,它的第 i 个元素是一支给定的股票在第 i 天的价格。

设计一个算法来计算你所能获取的最大利润。你最多可以完成 两笔 交易。

注意: 你不能同时参与多笔交易(你必须在再次购买前出售掉之前的股票)。

示例 1:

输入: [3,3,5,0,0,3,1,4]
输出: 6
解释: 在第 4 天(股票价格 = 0)的时候买入,在第 6 天(股票价格 = 3)的时候卖出,这笔交易所能获得利润 = 3-0 = 3 。
     随后,在第 7 天(股票价格 = 1)的时候买入,在第 8 天 (股票价格 = 4)的时候卖出,这笔交易所能获得利润 = 4-1 = 3 。
示例 2:

输入: [1,2,3,4,5]
输出: 4
解释: 在第 1 天(股票价格 = 1)的时候买入,在第 5 天 (股票价格 = 5)的时候卖出, 这笔交易所能获得利润 = 5-1 = 4 。   
     注意你不能在第 1 天和第 2 天接连购买股票,之后再将它们卖出。   
     因为这样属于同时参与了多笔交易,你必须在再次购买前出售掉之前的股票。
示例 3:

输入: [7,6,4,3,1] 
输出: 0 
解释: 在这个情况下, 没有交易完成, 所以最大利润为 0。































思路:
股票系列框架解法。

base case:
dp[-1][k][0] = dp[i][0][0] = 0
dp[-1][k][1] = dp[i][0][1] = -infinity

状态转移方程:
dp[i][k][0] = max(dp[i-1][k][0], dp[i-1][k][1] + prices[i])
dp[i][k][1] = max(dp[i-1][k][1], dp[i-1][k-1][0] - prices[i])



























code:
class Solution {
public:
    int maxProfit(vector<int>& prices) {
        int dp_i20 = 0, dp_i21 = INT_MIN, dp_i10 = 0, dp_i11 = INT_MIN;
        for(int i = 0; i < prices.size(); i++)
        {
            dp_i20 = max(dp_i20, dp_i21+prices[i]);
            dp_i21 = max(dp_i21, dp_i10-prices[i]);
            dp_i10 = max(dp_i10, dp_i11+prices[i]);
            dp_i11 = max(dp_i11, -prices[i]);
        }
        return dp_i20;
    }
};