给定一个仅包含数字 2-9 的字符串,返回所有它能表示的字母组合。

给出数字到字母的映射如下(与电话按键相同)。注意 1 不对应任何字母。

示例:

输入:"23"
输出:["ad", "ae", "af", "bd", "be", "bf", "cd", "ce", "cf"].

说明:
尽管上面的答案是按字典序排列的,但是你可以任意选择答案输出的顺





































思路:
若输入为 “236”,遍历 “236” 字符串。
首先,将 2 对应的字符放入 res 中,即 res = ["a", "b", "c"];
收到 3 时, res = ["ad","bd","cd","ae","be","ce","af","bf","cf"];
收到 6 时, res=["adm","bdm","cdm","aem","bem","cem","afm","bfm","cfm","adn","bdn","cdn","aen","ben","cen","afn","bfn","cfn","ado","bdo","cdo","aeo","beo","ceo","afo","bfo","cfo"]































code:
class Solution {
public:
    vector<string> letterCombinations(string digits) {
        vector<string> res;
        if(digits == "") return res;
        res.push_back("");
        map<char, string> map = {{'2', "abc"},
                                 {'3', "def"},
                                 {'4', "ghi"},
                                 {'5', "jkl"},
                                 {'6', "mno"},
                                 {'7', "pqrs"},
                                 {'8', "tuv"},
                                 {'9', "wxyz"}};
        for(int i = 0; i < digits.size(); i++)
        {
            if(digits[i] < '2' || digits[i] > '9') break;
            string cand = map[digits[i]];
            vector<string>temp;
            for(int j = 0; j < cand.size(); j++)
            {
                for(int k = 0; k < res.size(); k++)
                    temp.push_back(res[k]+cand[j]);
            }
            res = temp;
        }
        return res;
    }
};