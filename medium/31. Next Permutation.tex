实现获取下一个排列的函数,算法需要将给定数字序列重新排列成字典序中下一个更大的排列。

如果不存在下一个更大的排列,则将数字重新排列成最小的排列(即升序排列)。

必须原地修改,只允许使用额外常数空间。

以下是一些例子,输入位于左侧列,其相应输出位于右侧列。
1,2,3 → 1,3,2
3,2,1 → 1,2,3
1,1,5 → 1,5,1































思路:
从后往前找第一个 nums[i]<nums[i+1], index = i, 再从后往前找第一个 num[j] > nums[index],swap(nums[j], nums[index]),从index+1 位开始从小到大排列,即 sort(nums.begin()+index+1, nums.end());




























code:
class Solution {
public:
    void nextPermutation(vector<int>& nums) {
        int index = -1;
        for(int i = nums.size()-1; i >= 1; i--)
        {
            if(nums[i-1] < nums[i])
            {
                index = i-1; break;
            }
        }
        if(index == -1)
        {
            reverse(nums.begin(), nums.end());
            return;
        }
        for(int j = nums.size()-1; j >= 0; j--)
        {
            if(nums[j] > nums[index])
            {
                swap(nums[index], nums[j]); break;
            }
        }
        sort(nums.begin()+index+1, nums.end());
    }
};