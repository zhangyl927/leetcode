如果连续数字之间的差严格地在正数和负数之间交替,则数字序列称为摆动序列。第一个差(如果存在的话)可能是正数或负数。少于两个元素的序列也是摆动序列。

例如, [1,7,4,9,2,5] 是一个摆动序列,因为差值 (6,-3,5,-7,3) 是正负交替出现的。相反, [1,4,7,2,5] 和 [1,7,4,5,5] 不是摆动序列,第一个序列是因为它的前两个差值都是正数,第二个序列是因为它的最后一个差值为零。

给定一个整数序列,返回作为摆动序列的最长子序列的长度。 通过从原始序列中删除一些(也可以不删除)元素来获得子序列,剩下的元素保持其原始顺序。

示例 1:

输入: [1,7,4,9,2,5]
输出: 6 
解释: 整个序列均为摆动序列。
示例 2:

输入: [1,17,5,10,13,15,10,5,16,8]
输出: 7
解释: 这个序列包含几个长度为 7 摆动序列,其中一个可为[1,17,10,13,10,16,8]。
示例 3:

输入: [1,2,3,4,5,6,7,8,9]
输出: 2
进阶:
你能否用 O(n) 时间复杂度完成此题?



































思路:
1.创建两个数组 up 和down,遍历数组,
若 nums[i] > nums[i-1] 表示当前值为上升状态,up[i] = down[i-1]+1,up 值保持不变;
若 nums[i] < nums[i-1] 表示当前值为下降状态,down[i] = up[i-1]+1,up 值保持不变;
若 nums[i] == nums[i-1],up 和 down 的值保持不变;


优化:
仅使用 up 和 down 两个变量表示当前的值即可;



























优化代码:
class Solution {
public:
    int wiggleMaxLength(vector<int>& nums) {
        if(nums.size() == 0) return 0;
        int up, down;
        up = down = 1;
        for(int i = 1; i < nums.size(); i++)
        {
            if(nums[i] > nums[i-1]) up = down+1;
            else if(nums[i] < nums[i-1]) down = up+1;
        }
        return max(up, down);
    }
};





























code1:
class Solution {
public:
    int wiggleMaxLength(vector<int>& nums) {
        if(nums.size() == 0) return 0;
        int n = nums.size();
        vector<int> up(n);
        vector<int> down(n);
        up[0]=down[0]=1;
        for(int i = 1; i < n; i++)
        {
            if(nums[i] < nums[i-1])
            {
                up[i] = up[i-1];
                down[i] = up[i-1]+1;
            }
            else if(nums[i] > nums[i-1])
            {
                up[i] = down[i-1]+1;
                down[i] = down[i-1];
            }
            else
            {
                up[i] = up[i-1];
                down[i] = down[i-1];
            }
        }
        return max(up[n-1], down[n-1]);
    }
};