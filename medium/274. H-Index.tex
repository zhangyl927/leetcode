给定一位研究者论文被引用次数的数组(被引用次数是非负整数)。编写一个方法,计算出研究者的 h 指数。

h 指数的定义: “h 代表“高引用次数”(high citations),一名科研人员的 h 指数是指他(她)的 (N 篇论文中)至多有 h 篇论文分别被引用了至少 h 次。(其余的 N - h 篇论文每篇被引用次数不多于 h 次。)”

 

示例:

输入: citations = [3,0,6,1,5]
输出: 3 
解释: 给定数组表示研究者总共有 5 篇论文,每篇论文相应的被引用了 3, 0, 6, 1, 5 次。
     由于研究者有 3 篇论文每篇至少被引用了 3 次,其余两篇论文每篇被引用不多于 3 次,所以她的 h 指数是 3。
 

说明: 如果 h 有多种可能的值,h 指数是其中最大的那个。



























思路:
我们想象一个直方图,其中 xx 轴表示文章,yy 轴表示每篇文章的引用次数。如果将这些文章按照引用次数降序排序并在直方图上进行表示,那么直方图上的最大的正方形的边长 hh 就是我们所要求的 hh。



算法

首先我们将引用次数降序排序,在排完序的数组 citations 中,如果 icitations[i]>i,那么说明第 0 到 i 篇论文都有至少 i+1 次引用。因此我们只要找到最大的 i 满足 icitations[i]>i,那么 h 指数即为 i+1。例如:

i				 0	  1	   2	   3	    4   	 5	   6
引用次数			10	  9	   5	   3	    3	      2 	   1 
citations[i]>i		true	 true	 true	 false  false	false  false
其中最大的满足 citations[i]>i 的 i 值为 2,因此 h = i + 1 = 3h=i+1=3。






























code:
class Solution {
public:
    int hIndex(vector<int>& citations) {
    sort(citations.begin(), citations.end(), greater<int>());

    int hindex = 0;
    while(hindex<citations.size() && citations[hindex]>hindex)
        hindex++;
    return hindex;
}
};