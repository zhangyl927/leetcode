给定一个字符串S,检查是否能重新排布其中的字母,使得两相邻的字符不同。

若可行,输出任意可行的结果。若不可行,返回空字符串。

示例 1:

输入: S = "aab"
输出: "aba"
示例 2:

输入: S = "aaab"
输出: ""
注意:

S 只包含小写字母并且长度在[1, 500]区间内。
























思路:
若某个字母数多于一半,则 return “”;
每次 将频次最多的两个不同字母取出来;
注意 该字符出现的次数和它的 index 别搞混了!



























code:
class Solution {
public:
    string reorganizeString(string S) {
        vector<int> c(26, 0);
        vector<int> d(26, 0);
        for(int i = 0; i < S.size(); i++) c[S[i]-'a']++;
        string res;
        while(c != d)
        {
            int max = 0, maxindex1 = 0, maxindex2 = 0;
            for(int j = 0; j < 26; j++)
            {
                if(c[j] > max)
                {
                    max = c[j]; maxindex1 = j;
                }
            }
            int num = (accumulate(c.begin(), c.end(), 0)+1)/2;
            if( max > num ) return "";
            res += (maxindex1+'a');
            c[maxindex1]--;
            if(c == d) return res;
            max = 0;
            for(int j = 0; j < 26; j++)
            {
                if(j != maxindex1 && c[j] > max)
                {
                    max = c[j]; maxindex2 = j;
                }
            }
            res += (maxindex2+'a');
            c[maxindex2]--;
        }
        return res;
    }
};