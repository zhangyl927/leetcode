判断一个 9x9 的数独是否有效。只需要根据以下规则,验证已经填入的数字是否有效即可。

    数字 1-9 在每一行只能出现一次。
    数字 1-9 在每一列只能出现一次。
    数字 1-9 在每一个以粗实线分隔的 3x3 宫内只能出现一次。

上图是一个部分填充的有效的数独。

数独部分空格内已填入了数字,空白格用 '.' 表示。

示例 1:

输入:
[
  ["5","3",".",".","7",".",".",".","."],
  ["6",".",".","1","9","5",".",".","."],
  [".","9","8",".",".",".",".","6","."],
  ["8",".",".",".","6",".",".",".","3"],
  ["4",".",".","8",".","3",".",".","1"],
  ["7",".",".",".","2",".",".",".","6"],
  [".","6",".",".",".",".","2","8","."],
  [".",".",".","4","1","9",".",".","5"],
  [".",".",".",".","8",".",".","7","9"]
]
输出: true

示例 2:

输入:
[
  ["8","3",".",".","7",".",".",".","."],
  ["6",".",".","1","9","5",".",".","."],
  [".","9","8",".",".",".",".","6","."],
  ["8",".",".",".","6",".",".",".","3"],
  ["4",".",".","8",".","3",".",".","1"],
  ["7",".",".",".","2",".",".",".","6"],
  [".","6",".",".",".",".","2","8","."],
  [".",".",".","4","1","9",".",".","5"],
  [".",".",".",".","8",".",".","7","9"]
]
输出: false
解释: 除了第一行的第一个数字从 5 改为 8 以外,空格内其他数字均与 示例1 相同。
     但由于位于左上角的 3x3 宫内有两个 8 存在, 因此这个数独是无效的。

说明:

    一个有效的数独(部分已被填充)不一定是可解的。
    只需要根据以上规则,验证已经填入的数字是否有效即可。
    给定数独序列只包含数字 1-9 和字符 '.' 。
    给定数独永远是 9x9 形式的。





























思路:
参考如下code :
int isValidSudoku(char** board, int boardRowSize, int boardColSize) {
    int rows[9][9]={0}; //rows[5][0] means whether number 1('0'+1) in row 5 has appeared.
	int cols[9][9]={0}; //cols[3][8] means whether number 9('8'+1) in col 3 has appeared.
	int blocks[3][3][9]={0};//blocks[0][2][5] means whether number '6' in block 0,2 (row 0~2,col 6~8) has appeared.
	for(int r=0;r<9;r++)    //traverse board r,c
		for(int c=0;c<9;c++)
			if(board[r][c]!='.'){   //skip all number '.'
				int number=board[r][c]-'1'; //calculate the number's index(board's number minus 1)
				if(rows[r][number]++) return 0; //if the number has already appeared once, return false.
				if(cols[c][number]++) return 0;
				if(blocks[r/3][c/3][number]++) return 0;
			}
	return 1;
}


























code:
class Solution {
public:
    bool isValidSudoku(vector<vector<char>>& board) {
        int rows[9][9] = {0};
        int col[9][9] = {0};
        int blocks[3][3][9] = {0};
        for(int r = 0; r < 9; r++)
            for(int c = 0; c < 9; c++)
            {
                if(board[r][c] != '.')
                {
                    int number = board[r][c] - '1';
                    if(rows[r][number]++) return 0;
                    if(col[c][number]++) return 0;
                    if(blocks[r/3][c/3][number]++) return 0;
                }
            }
        return 1;
    }
};