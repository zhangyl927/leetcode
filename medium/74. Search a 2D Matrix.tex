编写一个高效的算法来判断 m x n 矩阵中,是否存在一个目标值。该矩阵具有如下特性:

    每行中的整数从左到右按升序排列。
    每行的第一个整数大于前一行的最后一个整数。

示例 1:

输入:
matrix = [
  [1,   3,  5,  7],
  [10, 11, 16, 20],
  [23, 30, 34, 50]
]
target = 3
输出: true

示例 2:

输入:
matrix = [
  [1,   3,  5,  7],
  [10, 11, 16, 20],
  [23, 30, 34, 50]
]
target = 13
输出: false





























思路:
1.先将矩阵放入容器 res 中,temp = res 后对 temp 排序,二分查找 target。仅当能找到 target 且temp == res,返回 true。
2. 两次二分法 分别获得 行和列。






























code2:
class Solution {
public:
    bool searchMatrix(vector<vector<int>>& matrix, int target) {
        if(matrix.size() == 0 || matrix[0].size() == 0) return false;
        int startrow = 0, endrow = matrix.size()-1, mid = 0;
        int startcol = 0, endcol = matrix[0].size()-1;
        int row = 0;
        // get the row
        while(startrow + 1 < endrow)
        {
            mid = startrow + (endrow-startrow)/2;
            if(matrix[mid][endcol] < target) startrow = mid;
            else endrow = mid;
        }
        if(matrix[startrow][endcol] >= target) row = startrow;
        else if(matrix[endrow][endcol] >= target) row = endrow;
        else return false;
        
        // get the line
        while(startcol + 1 < endcol)
        {
            mid = startcol + (endcol-startcol)/2;
            if(matrix[row][mid] < target) startcol = mid;
            else endcol = mid;
        }
        
        if(matrix[row][startcol] == target || matrix[row][endcol] == target) return true;
        else return false;
    }
};


























code:
class Solution {
public:
    bool searchMatrix(vector<vector<int>>& matrix, int target) {
        vector<int> res;
        for(int i = 0; i < matrix.size(); i++)
        {
            for(int j = 0; j < matrix[i].size(); j++)
            {
                res.push_back(matrix[i][j]);
            }
        }
        if(res.size() == 0) return false;
        vector<int> temp = res;
        sort(temp.begin(), temp.end());
        return binarysearch(0, temp.size()-1, target, temp) && (temp == res);
    }
    bool binarysearch(int begin, int end, int target, vector<int>& temp)
    {
        if(begin <= end)
        {
            int middle = begin + (end-begin)/2;
            if(target == temp[middle]) return true;
            else if(target > temp[middle]) return binarysearch(middle+1, end, target, temp);
            else return binarysearch(begin, middle-1, target, temp);
        }
        return false;
    }
};

优化 code:
class Solution {
public:
    bool searchMatrix(vector<vector<int>>& matrix, int target) {
        vector<int> res;
        for(int i = 0; i < matrix.size(); i++)
        {
            for(int j = 0; j < matrix[i].size(); j++)
            {
                res.push_back(matrix[i][j]);
            }
        }
        vector<int> temp = res;
        sort(temp.begin(), temp.end());
        if(temp.size() == 0 || temp[0] > target || temp.back() < target || temp != res) return false;
        
        // binarysearch
        int begin = 0, end = temp.size()-1;
        int mid = 0;
        while(begin <= end)
        {
            mid = begin + (end-begin)/2;
            if(target > temp[mid]) begin = mid+1;
            else if(target < temp[mid]) end = mid-1;
            else return true;
        }
        return false;
    }
};