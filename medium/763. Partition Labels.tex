字符串 S 由小写字母组成。我们要把这个字符串划分为尽可能多的片段,同一个字母只会出现在其中的一个片段。返回一个表示每个字符串片段的长度的列表。

示例 1:

输入: S = "ababcbacadefegdehijhklij"
输出: [9,7,8]
解释:
划分结果为 "ababcbaca", "defegde", "hijhklij"。
每个字母最多出现在一个片段中。
像 "ababcbacadefegde", "hijhklij" 的划分是错误的,因为划分的片段数较少。
注意:

S的长度在[1, 500]之间。
S只包含小写字母'a'到'z'。


























思路: 
 S = "a b a b c baca defegdehijhklij", 第一遍 遍历使用map 将每个字母最后的 index 存放起来
比如  8 5     7
第二次遍历,当 i = 第一段字母区间中字母的最大index时,截断;

















code:
class Solution {
public:
    vector<int> partitionLabels(string S) {
        vector<int> res;
        map<char,int> maps;
        for(int i=0; i<S.size(); i++)
            maps[S[i]] = i;
        int last = 0, start = 0;
        for(int i=0; i<S.size(); i++)
        {
            last = max(last, maps[S[i]]);
            if(last == i)
            {
                res.push_back(last-start+1);
                start = i+1;
            }
        }
        return res;
    }
};