给定两个整数,被除数 dividend 和除数 divisor。将两数相除,要求不使用乘法、除法和 mod 运算符。

返回被除数 dividend 除以除数 divisor 得到的商。

示例 1:

输入: dividend = 10, divisor = 3
输出: 3

示例 2:

输入: dividend = 7, divisor = -3
输出: -2

说明:

    被除数和除数均为 32 位有符号整数。
    除数不为 0。
    假设我们的环境只能存储 32 位有符号整数,其数值范围是 [−231,  231 − 1]。本题中,如果除法结果溢出,则返回 231 − 1。



























思路:
当 dividend 为 INT_MIN 时,若 divisor = -1,则结果为 2的31 次方 > INT_MAX;
long 为 8 个字节,防止计算过程中 出现溢出情况。
做一些预处理,保证计算过程中 dividend 和 divisor 都为 正数,引入了 sign 标志正负。
res 统计结果,如 dividend = 32, divisor = 3,第一次 res = 8;第二次 res = 8 + 2; res*sign 即为结果。 























code:
class Solution {
public:
    int divide(int dividend, int divisor) {
        if(dividend == INT_MIN)
        {
            if(divisor == -1) return INT_MAX;
            else if(divisor == 1) return INT_MIN;
        }
        long divid = (long)dividend;
        long divis = (long)divisor;
        int res = 0, sign = 1;
        if(divid < 0)
        {
            divid = -divid; sign = -sign;
        }
        if(divis < 0)
        {
            divis = -divis; sign = -sign;
        }
        while(divid >= divis)
        {
            int shift = 0;
            while(divid >= divis<<shift)
                shift++;
            res += (1 << (shift-1));
            divid -= (divis << (shift-1));
        }
        return res*sign;
    }
};