在上次打劫完一条街道之后和一圈房屋后,小偷又发现了一个新的可行窃的地区。这个地区只有一个入口,我们称之为“根”。 除了“根”之外,每栋房子有且只有一个“父“房子与之相连。一番侦察之后,聪明的小偷意识到“这个地方的所有房屋的排列类似于一棵二叉树”。 如果两个直接相连的房子在同一天晚上被打劫,房屋将自动报警。

计算在不触动警报的情况下,小偷一晚能够盗取的最高金额。

示例 1:

输入: [3,2,3,null,3,null,1]

     3
    / \
   2   3
    \   \ 
     3   1

输出: 7 
解释: 小偷一晚能够盗取的最高金额 = 3 + 3 + 1 = 7.
示例 2:

输入: [3,4,5,1,3,null,1]

     3
    / \
   4   5
  / \   \ 
 1   3   1

输出: 9
解释: 小偷一晚能够盗取的最高金额 = 4 + 5 = 9.
























思路:
https://www.youtube.com/watch?v=-i2BFAU25Zk
































code:
/**
 * Definition for a binary tree node.
 * struct TreeNode {
 *     int val;
 *     TreeNode *left;
 *     TreeNode *right;
 *     TreeNode(int x) : val(x), left(NULL), right(NULL) {}
 * };
 */
class Solution {
public:
    vector<int> robHelper(TreeNode* root)
    {
        if(!root) return {0, 0};
        vector<int> result(2);
        vector<int> left = robHelper(root->left);
        vector<int> right = robHelper(root->right);
        result[0] = max(left[0], left[1])+max(right[0], right[1]);
        result[1] = left[0]+right[0]+root->val;
        return result;
    }
    int rob(TreeNode* root) {
        vector<int> res = robHelper(root);
        return max(res[0], res[1]);
    }
};