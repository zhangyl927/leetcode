在一条环路上有 N 个加油站,其中第 i 个加油站有汽油 gas[i] 升。

你有一辆油箱容量无限的的汽车,从第 i 个加油站开往第 i+1 个加油站需要消耗汽油 cost[i] 升。你从其中的一个加油站出发,开始时油箱为空。

如果你可以绕环路行驶一周,则返回出发时加油站的编号,否则返回 -1。

说明: 

如果题目有解,该答案即为唯一答案。
输入数组均为非空数组,且长度相同。
输入数组中的元素均为非负数。
示例 1:

输入: 
gas  = [1,2,3,4,5]
cost = [3,4,5,1,2]

输出: 3

解释:
从 3 号加油站(索引为 3 处)出发,可获得 4 升汽油。此时油箱有 = 0 + 4 = 4 升汽油
开往 4 号加油站,此时油箱有 4 - 1 + 5 = 8 升汽油
开往 0 号加油站,此时油箱有 8 - 2 + 1 = 7 升汽油
开往 1 号加油站,此时油箱有 7 - 3 + 2 = 6 升汽油
开往 2 号加油站,此时油箱有 6 - 4 + 3 = 5 升汽油
开往 3 号加油站,你需要消耗 5 升汽油,正好足够你返回到 3 号加油站。
因此,3 可为起始索引。
示例 2:

输入: 
gas  = [2,3,4]
cost = [3,4,3]

输出: -1

解释:
你不能从 0 号或 1 号加油站出发,因为没有足够的汽油可以让你行驶到下一个加油站。
我们从 2 号加油站出发,可以获得 4 升汽油。 此时油箱有 = 0 + 4 = 4 升汽油
开往 0 号加油站,此时油箱有 4 - 3 + 2 = 3 升汽油
开往 1 号加油站,此时油箱有 3 - 3 + 3 = 3 升汽油
你无法返回 2 号加油站,因为返程需要消耗 4 升汽油,但是你的油箱只有 3 升汽油。
因此,无论怎样,你都不可能绕环路行驶一周。

























思路1:
将gas 和 cost 变成两倍,直接判断从 i~i+size 的能否符合就可以;

思路2:
开始的start_station = 0, total_tank 记录跑完所有一圈需要的油,current_tank 表示当前有多少油,若当前的油 < 0,则 start_station = i+1。



























code1:
class Solution {
public:
    int canCompleteCircuit(vector<int>& gas, vector<int>& cost) {
        vector<int> gas1 = gas;
        vector<int> cost1 = cost;
        for(auto g:gas)
            gas1.push_back(g);
        for(auto c:cost)
            cost1.push_back(c);
        
        for(int i = 0; i < gas.size(); i++)
        {
            if(canCompleteCircuitHelper(i, gas1, cost1)) return i;
        }
        return -1;
    }
    bool canCompleteCircuitHelper(int index, vector<int>& gas1, vector<int>& cost1)
    {
        int currentgas = 0;
        for(int i = 0; i < gas1.size()/2; i++)
        {
            currentgas += gas1[index+i];
            if(currentgas < cost1[index+i]) return false;
            currentgas -= cost1[index+i];
        }
        return currentgas >= 0;
    }
};


code2:
class Solution {
public:
    int canCompleteCircuit(vector<int>& gas, vector<int>& cost) {
        int current_tank = 0, total_tank = 0;
        int start_station = 0;
        for(int i = 0; i < gas.size(); i++)
        {
            total_tank += gas[i]-cost[i];
            current_tank += gas[i]-cost[i];
            if(current_tank < 0)
            {
                start_station = i+1;
                current_tank = 0;
            }
        }
        return (total_tank>=0)?start_station:-1;
    }
};