给定一个二叉树,它的每个结点都存放一个 0-9 的数字,每条从根到叶子节点的路径都代表一个数字。

例如,从根到叶子节点路径 1->2->3 代表数字 123。

计算从根到叶子节点生成的所有数字之和。

说明: 叶子节点是指没有子节点的节点。

示例 1:

输入: [1,2,3]
    1
   / \
  2   3
输出: 25
解释:
从根到叶子节点路径 1->2 代表数字 12.
从根到叶子节点路径 1->3 代表数字 13.
因此,数字总和 = 12 + 13 = 25.
示例 2:

输入: [4,9,0,5,1]
    4
   / \
  9   0
 / \
5   1
输出: 1026
解释:
从根到叶子节点路径 4->9->5 代表数字 495.
从根到叶子节点路径 4->9->1 代表数字 491.
从根到叶子节点路径 4->0 代表数字 40.
因此,数字总和 = 495 + 491 + 40 = 1026.

























思路:
类似 257 题解法。


























code:
/**
 * Definition for a binary tree node.
 * struct TreeNode {
 *     int val;
 *     TreeNode *left;
 *     TreeNode *right;
 *     TreeNode(int x) : val(x), left(NULL), right(NULL) {}
 * };
 */
class Solution {
public:
    void sumNumbers(TreeNode* Node, int& sum, vector<int>& sumTree)
    {
        if(!Node) return;
        sum = sum*10+Node->val;
        if(!Node->left && !Node->right) sumTree.push_back(sum);
        else
        {
            sumNumbers(Node->left, sum, sumTree);
            sumNumbers(Node->right, sum, sumTree);
        }
        sum = sum/10;
    }
    int sumNumbers(TreeNode* root) {
        int ret;
        int sum = 0;
        vector<int> sumTree;
        sumNumbers(root, sum, sumTree);
        //return sumTree[1];
        ret = accumulate(sumTree.begin(), sumTree.end(), 0);
        return ret;
    }
};