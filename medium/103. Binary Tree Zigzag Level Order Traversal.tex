给定一个二叉树,返回其节点值的锯齿形层次遍历。(即先从左往右,再从右往左进行下一层遍历,以此类推,层与层之间交替进行)。

例如:
给定二叉树 [3,9,20,null,null,15,7],

    3
   / \
  9  20
    /  \
   15   7

返回锯齿形层次遍历如下:

[
  [3],
  [20,9],
  [15,7]
]




















思路1:
同 102;

思路2:
层序遍历将值放入容器中,反转容器中奇数个容器中的值。




















优化 code:
/**
 * Definition for a binary tree node.
 * struct TreeNode {
 *     int val;
 *     TreeNode *left;
 *     TreeNode *right;
 *     TreeNode(int x) : val(x), left(NULL), right(NULL) {}
 * };
 */
class Solution {
public:
    vector<vector<int>> zigzagLevelOrder(TreeNode* root) {
        vector<vector<int>> ret;
        if(!root) return ret;
        int level = 0;
        queue<TreeNode*> mq;
        mq.push(root);
        while(!mq.empty())
        {
            ret.resize(level+1);
            int level_size = mq.size();
            for(int i = 0; i < level_size; i++)
            {
                TreeNode* p = mq.front();
                ret[level].push_back(p->val);
                mq.pop();

                if(p->left) mq.push(p->left);
                if(p->right) mq.push(p->right);
            }
            if(level%2 == 1) reverse(ret[level].begin(), ret[level].end());
            level++;
        }
        return ret;
    }
};



















code:
/**
 * Definition for a binary tree node.
 * struct TreeNode {
 *     int val;
 *     TreeNode *left;
 *     TreeNode *right;
 *     TreeNode(int x) : val(x), left(NULL), right(NULL) {}
 * };
 */
class Solution {
public:
    vector<vector<int>> zigzagLevelOrder(TreeNode* root) {
        vector<vector<int>> res;
        traversal(root, 0, res);
        for(int i = 0; i < res.size(); i++)
        {
            if(i % 2 == 1) reverse(res[i].begin(), res[i].end());
        }
        return res;
    }
    void traversal(TreeNode* Node, int depth, vector<vector<int>>& res)
    {
        if(!Node) return;
        if(res.size() == depth)
            res.resize(depth+1);
        res[depth].push_back(Node->val);
        traversal(Node->left, depth+1, res);
        traversal(Node->right, depth+1, res);
    }
};