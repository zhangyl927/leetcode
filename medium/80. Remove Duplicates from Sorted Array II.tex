给定一个排序数组,你需要在原地删除重复出现的元素,使得每个元素最多出现两次,返回移除后数组的新长度。

不要使用额外的数组空间,你必须在原地修改输入数组并在使用 O(1) 额外空间的条件下完成。

示例 1:

给定 nums = [1,1,1,2,2,3],

函数应返回新长度 length = 5, 并且原数组的前五个元素被修改为 1, 1, 2, 2, 3 。

你不需要考虑数组中超出新长度后面的元素。

示例 2:

给定 nums = [0,0,1,1,1,1,2,3,3],

函数应返回新长度 length = 7, 并且原数组的前五个元素被修改为 0, 0, 1, 1, 2, 3, 3 。

你不需要考虑数组中超出新长度后面的元素。

说明:

为什么返回数值是整数,但输出的答案是数组呢?

请注意,输入数组是以“引用”方式传递的,这意味着在函数里修改输入数组对于调用者是可见的。

你可以想象内部操作如下:

// nums 是以“引用”方式传递的。也就是说,不对实参做任何拷贝
int len = removeDuplicates(nums);

// 在函数里修改输入数组对于调用者是可见的。
// 根据你的函数返回的长度, 它会打印出数组中该长度范围内的所有元素。
for (int i = 0; i < len; i++) {
    print(nums[i]);
}






























思路:
思路同 26 题。
设置一个 index 和一个 lock,lock 和 index 从 2 开始。
当 !(nums[lock-1] == nums[lock-2] && nums[index] == nums[lock-1]) 时,
将 index 的值赋给 lock位置,同时lock 位置 +1: nums[lock++] = nums[index]



























code:
class Solution {
public:
    int removeDuplicates(vector<int>& nums) {
        if(nums.size() < 3) return nums.size();
        int index = 2, lock = 2;
        for(; index < nums.size(); index++)
        {
            if(!(nums[lock-1] == nums[lock-2] && nums[index] == nums[lock-1]))
                nums[lock++] = nums[index];
        }
        return lock;
    }
};