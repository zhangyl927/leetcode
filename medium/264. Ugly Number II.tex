编写一个程序,找出第 n 个丑数。

丑数就是只包含质因数 2, 3, 5 的正整数。

示例:

输入: n = 10
输出: 12
解释: 1, 2, 3, 4, 5, 6, 8, 9, 10, 12 是前 10 个丑数。

说明:  

    1 是丑数。
    n 不超过1690。































思路:
动态规划。
下一个数是前几个数中 某个数的 2 or 3 or 5 的倍数的最小值。注意可能某个数是2和3的倍数,如6,所以
if(current == 2*c[i]) i++;
if(current == 3*c[j]) j++;
if(current == 5*c[k]) k++;
这种形式是对的,不要用这种形式:
if(current == 2*c[i]) i++;
else if(current == 3*c[j]) j++;
else if(current == 5*c[k]) k++;
会造成重复。






























code:
class Solution {
public:
    int nthUglyNumber(int n) {
        vector<int> c(1,1);
        int i = 0, j = 0, k = 0, number = 1;
        int current = 1;
        while(number++ < n)
        {
            current = min(2*c[i],min(3*c[j], 5*c[k]));
            c.push_back(current);
            if(current == 2*c[i]) i++;
            if(current == 3*c[j]) j++;
            if(current == 5*c[k]) k++;
        }
        return current;
    }
};