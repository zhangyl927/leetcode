根据逆波兰表示法,求表达式的值。

有效的运算符包括 +, -, *, / 。每个运算对象可以是整数,也可以是另一个逆波兰表达式。

说明:

    整数除法只保留整数部分。
    给定逆波兰表达式总是有效的。换句话说,表达式总会得出有效数值且不存在除数为 0 的情况。

示例 1:

输入: ["2", "1", "+", "3", "*"]
输出: 9
解释: ((2 + 1) * 3) = 9

示例 2:

输入: ["4", "13", "5", "/", "+"]
输出: 6
解释: (4 + (13 / 5)) = 6

示例 3:

输入: ["10", "6", "9", "3", "+", "-11", "*", "/", "*", "17", "+", "5", "+"]
输出: 22
解释: 
  ((10 * (6 / ((9 + 3) * -11))) + 17) + 5
= ((10 * (6 / (12 * -11))) + 17) + 5
= ((10 * (6 / -132)) + 17) + 5
= ((10 * 0) + 17) + 5
= (0 + 17) + 5
= 17 + 5
= 22
































思路:
直接法。




























code:
class Solution {
public:
    int evalRPN(vector<string>& tokens) {
        stack<int> evalStack;
        for(auto c:tokens)
        {
            if(c == "+" || c == "-" || c == "*" || c == "/")
            {
                int result = 0;
                int temp1 = evalStack.top();
                evalStack.pop();
                int temp2 = evalStack.top();
                evalStack.pop();
                if(c == "+") result = temp2 + temp1;
                else if(c == "-") result = temp2 - temp1;
                else if(c == "*") result = temp2 * temp1;
                else result = temp2 / temp1;
                evalStack.push(result);
            }
            else evalStack.push(stoi(c));
        }
        return evalStack.top();
    }
};