给定一个二叉树,返回它的中序 遍历。

示例:

输入: [1,null,2,3]
   1
    \
     2
    /
   3

输出: [1,3,2]






















思路:
1.递归思想。
2.使用栈。

















code1:
/**
 * Definition for a binary tree node.
 * struct TreeNode {
 *     int val;
 *     TreeNode *left;
 *     TreeNode *right;
 *     TreeNode(int x) : val(x), left(NULL), right(NULL) {}
 * };
 */
class Solution {
public:
    vector<int> inorderTraversal(TreeNode* root) {
        vector<int> res;
        inordertraversal(root, res);
        return res;
    }
    void inordertraversal(TreeNode* Node, vector<int>& res)
    {
        if(!Node) return;
        inordertraversal(Node->left, res);
        res.push_back(Node->val);
        inordertraversal(Node->right, res);
    }
};








code2:
/**
 * Definition for a binary tree node.
 * struct TreeNode {
 *     int val;
 *     TreeNode *left;
 *     TreeNode *right;
 *     TreeNode(int x) : val(x), left(NULL), right(NULL) {}
 * };
 */
class Solution {
public:
    vector<int> inorderTraversal(TreeNode* root) {
        vector<int> result;
        stack<TreeNode*> myStack;
        TreeNode* current = root;
        
        while(!myStack.empty() || current != NULL)
        {
            if(current)
            {
                myStack.push(current);
                current = current->left;
            }
            else
            {
                result.push_back(myStack.top()->val);
                current = myStack.top()->right;
                myStack.pop();
            }
        }
        return result;
    }
};