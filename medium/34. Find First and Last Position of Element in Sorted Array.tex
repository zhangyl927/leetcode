给定一个按照升序排列的整数数组 nums,和一个目标值 target。找出给定目标值在数组中的开始位置和结束位置。

你的算法时间复杂度必须是 O(log n) 级别。

如果数组中不存在目标值,返回 [-1, -1]。

示例 1:

输入: nums = [5,7,7,8,8,10], target = 8
输出: [3,4]

示例 2:

输入: nums = [5,7,7,8,8,10], target = 6
输出: [-1,-1]


























思路:
1.直接法。首先从前往后找 第一次出现 target 的位置,再从后往前遍历一次找 最后一次出席那 target 的位置。时间复杂度最大为 O(n);
2.双指针法。前指针 front 从前往后遍历,后指针 rear 从后往前遍历。当找到时标志位设为 1。当两个标志位都为 1 时,则停止返回结果。





























code1:
class Solution {
public:
    vector<int> searchRange(vector<int>& nums, int target) {
        vector<int> res;
        int start = -1, end = -1;
        for(int i = 0; i < nums.size(); i++)
        {
            if(nums[i] == target)
            {
                start = i; break;
            }
        }
        for(int j = nums.size()-1; j >= 0; j--)
        {
            if(nums[j] == target)
            {
                end = j; break;
            }
        }
        res.push_back(start);
        res.push_back(end);
        return res;
    }
};



code2:
class Solution {
public:
    vector<int> searchRange(vector<int>& nums, int target) {
        if(nums.size() == 1 && nums[0] == target) return {0,0};
        vector<int> res;
        int start = -1, end = -1;
        int front = 0, rear = nums.size()-1, flag1 = 0, flag2 = 0;
        while(front <= rear)
        {
            if(nums[front] == target && flag1 == 0)
            {
                start = front; flag1 = 1;
            }
            if(nums[rear] == target && flag2 == 0)
            {
                end = rear; flag2 = 1;
            }
            if(flag1 && flag2) return {front, rear};
            if(flag1 == 0)
            {
                while(front + 1 < rear && nums[front] == nums[front+1]) front++;
                front++;
            }
            if(flag2 == 0)
            {
                while(rear - 1 > front && nums[rear] == nums[rear-1]) rear--;
                rear--;
            }
        }
        res.push_back(start); res.push_back(end);
        return res;
    }
};