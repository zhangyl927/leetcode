找到给定字符串(由小写字符组成)中的最长子串 T , 要求 T 中的每一字符出现次数都不少于 k 。输出 T 的长度。

示例 1:

输入:
s = "aaabb", k = 3

输出:
3

最长子串为 "aaa" ,其中 'a' 重复了 3 次。
示例 2:

输入:
s = "ababbc", k = 2

输出:
5

最长子串为 "ababb" ,其中 'a' 重复了 2 次, 'b' 重复了 3 次。

































思路:
切割方法:
两个指针 begin 和 end,用 end 遍历 s,当找到 s[end-'a'] 的次数 < k 时,取 max(ret, substr(begin, end-begin)中符合 > k 的子串长度); 


























code:
class Solution {
public:
    int longestSubstring(string s, int k) {
        int ret = 0;
        if(s.length() < k) return 0;
        vector<int> hash(26,0);
        bool fullString = true;
        
        for(auto c:s) hash[c-'a']++;
        for(int i = 0; i < s.length(); i++)
            if(hash[s[i]-'a'] < k)
            {
                fullString = false; break;
            }
        if(fullString) return s.length();
        
        int begin = 0, end = 0;
        while(end < s.length())
        {
            if(hash[s[end]-'a'] < k)
            {
                ret = max(ret, longestSubstring(s.substr(begin, end-begin), k));
                begin = end+1;
            }
            end++;
        }
        ret = max(ret, longestSubstring(s.substr(begin),k));
        return ret;
    }
};