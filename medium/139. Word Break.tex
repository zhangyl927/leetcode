给定一个非空字符串 s 和一个包含非空单词列表的字典 wordDict,判定 s 是否可以被空格拆分为一个或多个在字典中出现的单词。

说明:

拆分时可以重复使用字典中的单词。
你可以假设字典中没有重复的单词。
示例 1:

输入: s = "leetcode", wordDict = ["leet", "code"]
输出: true
解释: 返回 true 因为 "leetcode" 可以被拆分成 "leet code"。
示例 2:

输入: s = "applepenapple", wordDict = ["apple", "pen"]
输出: true
解释: 返回 true 因为 "applepenapple" 可以被拆分成 "apple pen apple"。
     注意你可以重复使用字典中的单词。
示例 3:

输入: s = "catsandog", wordDict = ["cats", "dog", "sand", "and", "cat"]
输出: false


































思路:
动态规划的思想;
将一个子串分解成 
for(i = 0 --> length)
	string left = s.substr(0,i), 
	string right = s.substr(i+1);
	判断 right 是否在字典中,且左边是否可分
































code:
class Solution {
public:
    bool wordBreak(string s, vector<string>& wordDict) {
        // Create a hashset of words for fast query.
        unordered_set<string> dict(wordDict.cbegin(), wordDict.cend());
        // Query the answer of the original string.
        return wordBreak(s, dict);
    }
    
    bool wordBreak(const string& s, const unordered_set<string>& dict) {
        // In memory, directly return.
        if(mem_.count(s)) return mem_[s];
        // Whole string is a word, memorize and return.
        if(dict.count(s)) return mem_[s]=true;
        // Try every break point.
        for(int j=1;j<s.length();j++) {
            const string left = s.substr(0,j);
            const string right = s.substr(j);
            // Find the solution for s.
            if(dict.count(right) && wordBreak(left, dict))
                return mem_[s]=true;
        }
        // No solution for s, memorize and return.
        return mem_[s]=false;
    }
private:
    unordered_map<string, bool> mem_;
};