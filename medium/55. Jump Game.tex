给定一个非负整数数组,你最初位于数组的第一个位置。

数组中的每个元素代表你在该位置可以跳跃的最大长度。

判断你是否能够到达最后一个位置。

示例 1:

输入: [2,3,1,1,4]
输出: true
解释: 从位置 0 到 1 跳 1 步, 然后跳 3 步到达最后一个位置。
示例 2:

输入: [3,2,1,0,4]
输出: false
解释: 无论怎样,你总会到达索引为 3 的位置。但该位置的最大跳跃长度是 0 , 所以你永远不可能到达最后一个位置。





















思路:
1.回溯法  超时
2.自顶向下动态规划法 超时
3.自底向上动态规划法 相较3,不需要栈
自底向上和自顶向下动态规划的区别就是消除了回溯,在实际使用中,自底向下的方法有更好的时间效率因为我们不再需要栈空间,可以节省很多缓存开销。更重要的事,这可以让之后更有优化的空间。回溯通常是通过反转动态规划的步骤来实现的。

这是由于我们每次只会向右跳动,意味着如果我们从右边开始动态规划,每次查询右边节点的信息,都是已经计算过了的,不再需要额外的递归开销,因为我们每次在 memo 表中都可以找到结果。


4.贪心法
































code4:
class Solution {
public:
    bool canJump(vector<int>& nums) {
        int lastp = nums.size()-1;
        for(int i = nums.size()-1; i >= 0; i--)
        {
            if(i+nums[i] >= lastp)
                lastp = i;
        }
        return lastp == 0;
    }
};