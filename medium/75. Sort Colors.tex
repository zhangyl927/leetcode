给定一个包含红色、白色和蓝色,一共 n 个元素的数组,原地对它们进行排序,使得相同颜色的元素相邻,并按照红色、白色、蓝色顺序排列。

此题中,我们使用整数 0、 1 和 2 分别表示红色、白色和蓝色。

注意:
不能使用代码库中的排序函数来解决这道题。

示例:

输入: [2,0,2,1,1,0]
输出: [0,0,1,1,2,2]

进阶:

    一个直观的解决方案是使用计数排序的两趟扫描算法。
    首先,迭代计算出0、1 和 2 元素的个数,然后按照0、1、2的排序,重写当前数组。
    你能想出一个仅使用常数空间的一趟扫描算法吗?





























思路:
荷兰国旗问题









update:
设两个 指针 cfirst 和 clast,idx 从前往后扫描,
当 nums[idx] == 1 时直接往前移动;
当 nums[idx] == 0, swap(当前值,0的下一个位置),并且 idx 往前移动,cfirst 往前移动;
当 nums[idx] == 2, swap(当前值,2), idx 保持不变,clast 往后移动。

class Solution {
public:
    void sortColors(vector<int>& nums) {
        int size = nums.size(), cfirst = 0, clast = size-1, idx = 0;
        while (idx <= clast) {
            if(nums[idx] == 1) idx++;
            else if(nums[idx] == 0) swap(nums[idx++], nums[cfirst++]);
            else if(nums[idx] == 2) swap(nums[idx], nums[clast--]); 
        }
    }
};



















code:
class Solution {
public:
    void sortColors(vector<int>& nums) {
        int count1 = 0, count2 = 0;
        int n = nums.size();
        for(int i = 0; i <= n-1-count1; i++)
        {
            while(nums[i] > 0 && i <= n-1-count1)
            {
                if(nums[i] == 2)
                {
                    swap(nums[i], nums[n-1-count2]); count2++; count1++;
                    if(count1 > count2) count1--;
                }
                else if(nums[i] == 1)
                {
                    swap(nums[i], nums[n-1-count1]); count1++;
                }
            }
        }
    }
};