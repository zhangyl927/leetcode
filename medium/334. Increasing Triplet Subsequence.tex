给定一个未排序的数组,判断这个数组中是否存在长度为 3 的递增子序列。

数学表达式如下:

如果存在这样的 i, j, k,  且满足 0 ≤ i < j < k ≤ n-1,
使得 arr[i] < arr[j] < arr[k] ,返回 true ; 否则返回 false 。
说明: 要求算法的时间复杂度为 O(n),空间复杂度为 O(1) 。

示例 1:

输入: [1,2,3,4,5]
输出: true
示例 2:

输入: [5,4,3,2,1]
输出: false


























思路:
在遍历过程中有 3 中情况:case1, case2, case3.
当前值为 cur 时,
case 1: cur < min1,此时 min1 替换为 cur;
case 2: cur > min1 && cur < min2,此时 min2 替换为 cur;
case 3: cur > min2,此时找到了递增的三元子序列;






























code:
class Solution {
public:
    bool increasingTriplet(vector<int>& nums) {
        if(nums.size() <= 2) return false;
        int min1 = INT_MAX, min2 = INT_MAX;
        for(int i = 0; i < nums.size(); i++)
        {
            int cur = nums[i];
            // case 3;
            if(cur > min2) return true;
            // case 1
            else if(cur < min1) min1 = cur;
            // case 2
            else if(cur>min1 && cur<min2) min2 = cur;
        }
        return false;
    }
};
