给定一个链表,旋转链表,将链表每个节点向右移动 k 个位置,其中 k 是非负数。

示例 1:

输入: 1->2->3->4->5->NULL, k = 2
输出: 4->5->1->2->3->NULL
解释:
向右旋转 1 步: 5->1->2->3->4->NULL
向右旋转 2 步: 4->5->1->2->3->NULL

示例 2:

输入: 0->1->2->NULL, k = 4
输出: 2->0->1->NULL
解释:
向右旋转 1 步: 2->0->1->NULL
向右旋转 2 步: 1->2->0->NULL
向右旋转 3 步: 0->1->2->NULL
向右旋转 4 步: 2->0->1->NULL

























思路:
首先计算链表中元素的个数 count,k = k % count;
循环 k 次,每次都在开头插入一个 值为 0 的节点dummy,遍历指针找到最后一位,删除最后一个节点,并将最后一个节点的值赋给 dummy。






























code:
/**
 * Definition for singly-linked list.
 * struct ListNode {
 *     int val;
 *     ListNode *next;
 *     ListNode(int x) : val(x), next(NULL) {}
 * };
 */
class Solution {
public:
    ListNode* rotateRight(ListNode* head, int k) {
        if(head == NULL) return NULL;
        ListNode* p = head;
        int count = 0;
        while(p)
        {
            p = p->next; count++;
        }
        if(count == 1) return head;
        if(k > count) k %= count;
        int c = 0;
        for(int i = 0; i < k; i++)
        {
            ListNode* dummy = new ListNode(0);
            dummy->next = head;
            p = head;
            c = 0;
            while(p)
            {
                c++;
                if(c == count-1) 
                {
                    dummy->val = p->next->val;
                    p->next = NULL;  break;
                }
                p = p->next;
            }
            head = dummy;
        }
        return head;
    }
};