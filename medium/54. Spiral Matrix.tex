给定一个包含 m x n 个元素的矩阵(m 行, n 列),请按照顺时针螺旋顺序,返回矩阵中的所有元素。

示例 1:

输入:
[
 [ 1, 2, 3 ],
 [ 4, 5, 6 ],
 [ 7, 8, 9 ]
]
输出: [1,2,3,6,9,8,7,4,5]

示例 2:

输入:
[
  [1, 2, 3, 4],
  [5, 6, 7, 8],
  [9,10,11,12]
]
输出: [1,2,3,4,8,12,11,10,9,5,6,7]

































思路:
eg1:
12   3
  5  6
4
7   89
上图扫描顺序为 12,36,98,74, 剩余为 5
eg2:
1  2  3     4
   6  7     8
5
9    10 11 12
上图扫描顺序为 123, 48,121110,95,  剩余为67
eg3:
1   2       3
            6
4   5       9
7   8
10     11  12
上图 扫描顺序为 12, 369, 1211,1074, 剩余为 58
以上三种情况包含了所有剩余的情况。

根据扫描特点,我们可以设置 4 个点 left = 0, right = matrix[0].size()-1, top = 0, bottom = natrix.size()-1;




























code:
class Solution {
public:
    vector<int> spiralOrder(vector<vector<int>>& matrix) {
        if(matrix.size() == 0) return {};
        int left = 0, right = matrix[0].size()-1;
        int top = 0, bottom = matrix.size()-1;
        vector<int> res;
        while(left < right && top < bottom)
        {
            for(int i = left; i < right; i++) res.push_back(matrix[top][i]);
            for(int i = top; i < bottom; i++) res.push_back(matrix[i][right]);
            for(int i = right; i > left; i--) res.push_back(matrix[bottom][i]);
            for(int i = bottom; i > top; i--) res.push_back(matrix[i][left]);
            left++; right--;
            top++; bottom--;
        }
        if(left == right)
        {
            for(int i = top; i <= bottom; i++) res.push_back(matrix[i][left]);
        }
        else if(top == bottom)
        {
            for(int i = left; i <= right; i++) res.push_back(matrix[top][i]);
        }
        return res;
    }
};