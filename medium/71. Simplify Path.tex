以 Unix 风格给出一个文件的绝对路径,你需要简化它。或者换句话说,将其转换为规范路径。

在 Unix 风格的文件系统中,一个点(.)表示当前目录本身;此外,两个点 (..) 表示将目录切换到上一级(指向父目录);两者都可以是复杂相对路径的组成部分。更多信息请参阅:Linux / Unix中的绝对路径 vs 相对路径

请注意,返回的规范路径必须始终以斜杠 / 开头,并且两个目录名之间必须只有一个斜杠 /。最后一个目录名(如果存在)不能以 / 结尾。此外,规范路径必须是表示绝对路径的最短字符串。

 

示例 1:

输入:"/home/"
输出:"/home"
解释:注意,最后一个目录名后面没有斜杠。

示例 2:

输入:"/../"
输出:"/"
解释:从根目录向上一级是不可行的,因为根是你可以到达的最高级。

示例 3:

输入:"/home//foo/"
输出:"/home/foo"
解释:在规范路径中,多个连续斜杠需要用一个斜杠替换。

示例 4:

输入:"/a/./b/../../c/"
输出:"/c"

示例 5:

输入:"/a/../../b/../c//.//"
输出:"/c"

示例 6:

输入:"/a//b////c/d//././/.."
输出:"/a/b/c"






























思路:
使用堆栈的思想将字符串保存起来,若来的字符串是“..” 且栈不为空,则弹出最后一个;
将容器作为栈实现这种思想,可以省去弹出时逆序;

































code:
class Solution {
public:
    string simplifyPath(string path) {
        string tmp, res;
        vector<string> s1;
        stringstream ss(path);
        while(getline(ss, tmp, '/'))
        {
            if(tmp=="" || tmp==".") continue;
            else if(tmp=="..")
            {
                if(!s1.empty())s1.pop_back();
                else continue;
            }
            else s1.push_back(tmp);
        }
        for(auto c:s1) res += "/"+c;
        return (s1.empty())?"/":res;
    }
};