罗马数字包含以下七种字符: I, V, X, L,C,D 和 M。

字符          数值
I             1
V             5
X             10
L             50
C             100
D             500
M             1000

例如, 罗马数字 2 写做 II ,即为两个并列的 1。12 写做 XII ,即为 X + II 。 27 写做  XXVII, 即为 XX + V + II 。

通常情况下,罗马数字中小的数字在大的数字的右边。但也存在特例,例如 4 不写做 IIII,而是 IV。数字 1 在数字 5 的左边,所表示的数等于大数 5 减小数 1 得到的数值 4 。同样地,数字 9 表示为 IX。这个特殊的规则只适用于以下六种情况:

    I 可以放在 V (5) 和 X (10) 的左边,来表示 4 和 9。
    X 可以放在 L (50) 和 C (100) 的左边,来表示 40 和 90。 
    C 可以放在 D (500) 和 M (1000) 的左边,来表示 400 和 900。

给定一个整数,将其转为罗马数字。输入确保在 1 到 3999 的范围内。

示例 1:

输入: 3
输出: "III"

示例 2:

输入: 4
输出: "IV"

示例 3:

输入: 9
输出: "IX"

示例 4:

输入: 58
输出: "LVIII"
解释: L = 50, V = 5, III = 3.

示例 5:

输入: 1994
输出: "MCMXCIV"
解释: M = 1000, CM = 900, XC = 90, IV = 4.




























思路:
贪心




































code:
class Solution {
public:
    string intToRoman(int num) {
        string res;
        vector<int> val{1000,900,500,400,100,90,50,40,10,9,5,4,1};
        vector<string> lm{"M","CM","D","CD","C","XC","L","XL","X","IX","V","IV","I"};
        for(int i = 0; i < val.size(); i++)
        {
            while(num >= val[i])
            {
                num -= val[i]; res += lm[i];
            }
        }
        return res;
    }
};