给定一个以字符串表示的非负整数 num,移除这个数中的 k 位数字,使得剩下的数字最小。

注意:

num 的长度小于 10002 且 ≥ k。
num 不会包含任何前导零。
示例 1 :

输入: num = "1432219", k = 3
输出: "1219"
解释: 移除掉三个数字 4, 3, 和 2 形成一个新的最小的数字 1219。
示例 2 :

输入: num = "10200", k = 1
输出: "200"
解释: 移掉首位的 1 剩下的数字为 200. 注意输出不能有任何前导零。
示例 3 :

输入: num = "10", k = 2
输出: "0"
解释: 从原数字移除所有的数字,剩余为空就是0。


































思路:
使用栈将数字保存起来,当有 132 这种情况时,删除 3 即可;
考虑 123456789 这种情况,应该从后往前删;
考虑 10200 这种情况,应该去除前面的 0;


































code:
class Solution {
public:
    string removeKdigits(string num, int k) {
        if(k >= num.size()) return "0";
        int size = num.size();
        stack<char> st;
        string ret;
        for(int i = 0; i < size; i++)
        {
            while(!st.empty() && st.top()>num[i] && k>0)
            {
                st.pop(); k--;
            }
            st.push(num[i]);
        }
        while(k--) st.pop();
        while(!st.empty())
        {
            ret += st.top(); st.pop();
        }
        while(ret.size()>0 && ret[ret.size()-1] == '0') ret.pop_back();
        if(ret == "") return "0";
        reverse(ret.begin(), ret.end());
        return ret;
    }
};