实现 pow(x, n) ,即计算 x 的 n 次幂函数。

示例 1:

输入: 2.00000, 10
输出: 1024.00000

示例 2:

输入: 2.10000, 3
输出: 9.26100

示例 3:

输入: 2.00000, -2
输出: 0.25000
解释: 2-2 = 1/22 = 1/4 = 0.25

说明:

    -100.0 < x < 100.0
    n 是 32 位有符号整数,其数值范围是 [−231, 231 − 1] 。
























思路:
参考:
1.https://www.youtube.com/watch?v=yEQq3t3T_J0
2.https://leetcode.com/problems/powx-n/discuss/19578/Non-recursive-C%2B%2B-log(n)-solution
本题本质上是一种二分法。
如 pow(2, 9) = 2*pow(4, 4) = 2*pow(16,2) = 256;
res		 x 		n 
1 		 2  		9
1*2  		 4   		4
1*2  		 16		2
1*2 		 256  	1
1*2*256
注意处理 INT_MIN:if(n < 0) return 1/(x*myPow(x, -(n+1)));


























code:
class Solution {
public:
    double myPow(double x, int n) {
        if(n == 0 || x == 1) return 1;
        if(n == 1) return x;
        if(n < 0) return 1/(x*myPow(x, -(n+1)));
        double res = 1;
        while(n)
        {
            if(n & 1) res *= x;
            x *= x;
            n >>= 1;
        }
        return res;
    }
};
