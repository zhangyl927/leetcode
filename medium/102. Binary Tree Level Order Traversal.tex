给定一个二叉树,返回其按层次遍历的节点值。 (即逐层地,从左到右访问所有节点)。

例如:
给定二叉树: [3,9,20,null,null,15,7],

    3
   / \
  9  20
    /  \
   15   7
返回其层次遍历结果:

[
  [3],
  [9,20],
  [15,7]
]

























思路1:
用队列实现。
每次记录同一层的元素个数,遍历每一层,将当前层的元素都 pop 出队列,将下一层元素放入队列中。

思路2:
递归实现。


























code1:
/**
 * Definition for a binary tree node.
 * struct TreeNode {
 *     int val;
 *     TreeNode *left;
 *     TreeNode *right;
 *     TreeNode(int x) : val(x), left(NULL), right(NULL) {}
 * };
 */
class Solution {
public:
    vector<vector<int>> levelOrder(TreeNode* root) {
        if(!root) return {};
        vector<vector<int>> ret;
        queue<TreeNode*> mq;
        mq.push(root);
        int level = 0;
        while(!mq.empty())
        {
            ret.resize(level+1);
            int level_length = mq.size();
            for(int i = 0; i < level_length; i++)
            {
                TreeNode* p = mq.front();
                ret[level].push_back(p->val);
                mq.pop();

                if(p->left != NULL) mq.push(p->left);
                if(p->right != NULL) mq.push(p->right);
            }
            level++;
        }
        return ret;
    }
};





code2:
/**
 * Definition for a binary tree node.
 * struct TreeNode {
 *     int val;
 *     TreeNode *left;
 *     TreeNode *right;
 *     TreeNode(int x) : val(x), left(NULL), right(NULL) {}
 * };
 */
class Solution {
public:
    void levelOrder(TreeNode* Node, vector<vector<int>>& ret, int depth)
    {
        if(!Node) return;
        if(ret.size() == depth)
        {
            ret.resize(depth+1);
        }
        ret[depth].push_back(Node->val);
        if(Node->left) levelOrder(Node->left, ret, depth+1);
        if(Node->right) levelOrder(Node->right, ret, depth+1);
    }
    vector<vector<int>> levelOrder(TreeNode* root) {
        vector<vector<int>> ret;
        levelOrder(root, ret, 0);
        return ret;
    }
};