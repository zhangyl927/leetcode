你正在和你的朋友玩 猜数字(Bulls and Cows)游戏:你写下一个数字让你的朋友猜。每次他猜测后,你给他一个提示,告诉他有多少位数字和确切位置都猜对了(称为“Bulls”, 公牛),有多少位数字猜对了但是位置不对(称为“Cows”, 奶牛)。你的朋友将会根据提示继续猜,直到猜出秘密数字。

请写出一个根据秘密数字和朋友的猜测数返回提示的函数,用 A 表示公牛,用 B 表示奶牛。

请注意秘密数字和朋友的猜测数都可能含有重复数字。

示例 1:

输入: secret = "1807", guess = "7810"

输出: "1A3B"

解释: 1 公牛和 3 奶牛。公牛是 8,奶牛是 0, 1 和 7。
示例 2:

输入: secret = "1123", guess = "0111"

输出: "1A1B"

解释: 朋友猜测数中的第一个 1 是公牛,第二个或第三个 1 可被视为奶牛。
说明: 你可以假设秘密数字和朋友的猜测数都只包含数字,并且它们的长度永远相等。

























思路:
遍历 secret,if secret[i] == guess[i], A++;
		  else 将 secret[i] 保存在 cnt 中
遍历 guess,当 secret[j] != guess[j] 时, 若 secret[j] 在 cnt 中,则 cnt 中当前字符的数量 -1,且 B++;



























code:
class Solution {
public:
    string getHint(string secret, string guess) {
        int n = secret.size();
        if(n == 0) return "0A0B";
        vector<int> cnt(10, 0);
        int A = 0, B = 0;
        for(int i = 0; i < n; i++)
        {
            if(secret[i] == guess[i]) A++;
            else cnt[secret[i]-'0']++;
        }
        for(int j = 0; j < n; j++)
        {
            if(secret[j]!=guess[j] && cnt[guess[j]-'0']>0)
            {
                cnt[guess[j]-'0']--; B++;
            }
        }
        return to_string(A)+"A"+to_string(B)+"B";
    }
};