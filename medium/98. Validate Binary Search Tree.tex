给定一个二叉树,判断其是否是一个有效的二叉搜索树。

假设一个二叉搜索树具有如下特征:

    节点的左子树只包含小于当前节点的数。
    节点的右子树只包含大于当前节点的数。
    所有左子树和右子树自身必须也是二叉搜索树。

示例 1:

输入:
    2
   / \
  1   3
输出: true

示例 2:

输入:
    5
   / \
  1   4
     / \
    3   6
输出: false
解释: 输入为: [5,1,4,null,null,3,6]。
     根节点的值为 5 ,但是其右子节点值为 4 。



















思路1:
同 94。



思路2:
中序遍历结果存放在容器中,若不是升序的,返回 false。

















code1:
/**
 * Definition for a binary tree node.
 * struct TreeNode {
 *     int val;
 *     TreeNode *left;
 *     TreeNode *right;
 *     TreeNode(int x) : val(x), left(NULL), right(NULL) {}
 * };
 */
class Solution {
public:
    bool isValidBST(TreeNode* root) {
        vector<int> ret;
        stack<TreeNode*> st;
        TreeNode* cur = root;
        while(cur!=NULL || !st.empty())
        {
            while(cur != NULL)
            {
                st.push(cur);
                cur = cur->left;
            }
            cur = st.top();
            if(!ret.empty() && cur->val<=ret.back()) return false;
            ret.push_back(cur->val);
            st.pop();
            cur = cur->right;
        }
        return true;
    }
};




























code2:
/**
 * Definition for a binary tree node.
 * struct TreeNode {
 *     int val;
 *     TreeNode *left;
 *     TreeNode *right;
 *     TreeNode(int x) : val(x), left(NULL), right(NULL) {}
 * };
 */
class Solution {
public:
    bool isValidBST(TreeNode* root) {
        vector<int> res;
        inorderTraversal(root, res);
        for(int i = 1; i < res.size(); i++)
        {
            if(res[i] <= res[i-1]) return false;
        }
        return true;
    }
    void inorderTraversal(TreeNode* Node, vector<int>& res)
    {
        if(!Node) return;
        inorderTraversal(Node->left, res);
        res.push_back(Node->val);
        inorderTraversal(Node->right, res);
    }
};