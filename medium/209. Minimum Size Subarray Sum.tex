给定一个含有 n 个正整数的数组和一个正整数 s ,找出该数组中满足其和 ≥ s 的长度最小的连续子数组。如果不存在符合条件的连续子数组,返回 0。

示例: 

输入: s = 7, nums = [2,3,1,2,4,3]
输出: 2
解释: 子数组 [4,3] 是该条件下的长度最小的连续子数组。
进阶:

如果你已经完成了O(n) 时间复杂度的解法, 请尝试 O(n log n) 时间复杂度的解法。




























思路:
遍历求和,left 为左指针,i 遍历,第一次得到 sum >= s 时,将长度 i-left+1 保存,更新 sum = sum-nums[left],left 右移。


























code:
class Solution {
public:
    int minSubArrayLen(int s, vector<int>& nums) {
        int sum = 0, left = 0;
        int res = INT_MAX;
        for(int i = 0; i < nums.size(); i++)
        {
            sum += nums[i];
            while(sum >= s)
            {
                res = min(res, i-left+1);
                sum -= nums[left];
                left++;
            }
        }
        return (res==INT_MAX)?0:res;
    }
};
