编写一段程序来查找第 n 个超级丑数。

超级丑数是指其所有质因数都是长度为 k 的质数列表 primes 中的正整数。

示例:

输入: n = 12, primes = [2,7,13,19]
输出: 32 
解释: 给定长度为 4 的质数列表 primes = [2,7,13,19],前 12 个超级丑数序列为:[1,2,4,7,8,13,14,16,19,26,28,32] 。
说明:

1 是任何给定 primes 的超级丑数。
 给定 primes 中的数字以升序排列。
0 < k ≤ 100, 0 < n ≤ 106, 0 < primes[i] < 1000 。
第 n 个超级丑数确保在 32 位有符整数范围内。





















思路:
在 264 题基础上进行修改,264题如下:

编写一个程序,找出第 n 个丑数。

丑数就是只包含质因数 2, 3, 5 的正整数。

示例:

输入: n = 10
输出: 12
解释: 1, 2, 3, 4, 5, 6, 8, 9, 10, 12 是前 10 个丑数。


将 2,3,5 用一个 vector 装入,即是 vector<int>& primes;
2,3,5 对应的 i,j,k 为 与 primes 大小相同的 vector<int> vk(k, 0) 来表示。
vector<int> res 仍然存放丑数的集合。