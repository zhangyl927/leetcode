给定两个以字符串形式表示的非负整数 num1 和 num2,返回 num1 和 num2 的乘积,它们的乘积也表示为字符串形式。

示例 1:

输入: num1 = "2", num2 = "3"
输出: "6"
示例 2:

输入: num1 = "123", num2 = "456"
输出: "56088"
说明:

num1 和 num2 的长度小于110。
num1 和 num2 只包含数字 0-9。
num1 和 num2 均不以零开头,除非是数字 0 本身。
不能使用任何标准库的大数类型(比如 BigInteger)或直接将输入转换为整数来处理。




















思路:
使用乘法计算。






















code:
class Solution {
public:
    string multiply(string num1, string num2) {
        if(num1.length()==0 || num2.length()==0) return "0";
        int n1 = num1.size(), n2 = num2.size();
        vector<int> result(n1+n2);
        for(int i = n1-1; i >= 0; i--)
        {
            for(int j = n2-1; j >= 0; j--)
            {
                int mul = (num1[i]-'0')*(num2[j]-'0');
                int posLow = i+j+1;
                int posHigh = i+j;
                mul += result[posLow];
                result[posLow] = mul%10;
                result[posHigh] += mul/10;
            }
        }
        string sb;
        for(auto res:result)
        {
            if(!(sb.length()==0 && res==0)) sb += char(res+'0');
        }
        return (sb.empty())?"0":sb;
    }
};