一个机器人位于一个 m x n 网格的左上角 (起始点在下图中标记为“Start” )。

机器人每次只能向下或者向右移动一步。机器人试图达到网格的右下角(在下图中标记为“Finish”)。

问总共有多少条不同的路径?

例如,上图是一个7 x 3 的网格。有多少可能的路径?

说明:m 和 n 的值均不超过 100。

示例 1:

输入: m = 3, n = 2
输出: 3
解释:
从左上角开始,总共有 3 条路径可以到达右下角。
1. 向右 -> 向右 -> 向下
2. 向右 -> 向下 -> 向右
3. 向下 -> 向右 -> 向右

示例 2:

输入: m = 7, n = 3
输出: 28


























思路:
动态规划。
1.状态变量
2.初始化
3.转移方程
4.返回结果。
参考: https://www.youtube.com/watch?v=O9GhDaafmmo




























code:
class Solution {
public:
    int uniquePaths(int m, int n) {
        vector<vector<int>> dp(m, vector<int>(n));
        for(int i = 0; i < m; i++)
            dp[i][0] = 1;
        for(int j = 0; j < n; j++)
            dp[0][j] = 1;
        for(int i = 1; i < m; i++)
        {
            for(int j = 1; j < n; j++)
            {
                dp[i][j] = dp[i-1][j] + dp[i][j-1];
            }
        }
        return dp[m-1][n-1];
    }
};