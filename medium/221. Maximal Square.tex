在一个由 0 和 1 组成的二维矩阵内,找到只包含 1 的最大正方形,并返回其面积。

示例:

输入: 

1 0 1 0 0
1 0 1 1 1
1 1 1 1 1
1 0 0 1 0

输出: 4
























思路:
dp;
输入:
			1 0 1 0 0
			1 0 1 1 1
			1 1 1 1 1
			1 0 0 1 0

dp table: 	   0 0 0 0 0 0
		   0 1 0 1 0 0
		   0 1 0 1 1 1
		   0 1 1 1 2 1
		   0 1 0 0 1 0

choice the max; return max*max;




优化 dp 空间为 2 vector:
仅使用两行,一行 current 存放当前的值,pre 存放前一行的值;




优化 dp 空间为 1 vector:
curl 存放上一行的值,pre 存放当前值的左上角的值;

输入:
			1 0 1 0 0
			1 0 1 1 1
			1 1 1 1 1
			1 0 0 1 0

遍历第一行得到 curl: 1 0 1 0 0
遍历第二行时,pre 存放左上角的值,
 cur[j]  =  min(pre, min(cur[j],             cur[j - 1]))       + 1;
当前的 j           上一行的 j,还未覆盖   当前行的j-1,已经覆盖








优化 1 vector code:
class Solution {
public:
    int maximalSquare(vector<vector<char>>& matrix) {
        if (matrix.empty()) {
            return 0;
        }
        int m = matrix.size(), n = matrix[0].size(), sz = 0, pre;
        vector<int> cur(n, 0);
        for (int i = 0; i < m; i++) {
            for (int j = 0; j < n; j++) {
                int temp = cur[j];
                if (!i || !j || matrix[i][j] == '0') {
                    cur[j] = matrix[i][j] - '0';
                } else {
                    cur[j] = min(pre, min(cur[j], cur[j - 1])) + 1;
                }
                sz = max(cur[j], sz);
                pre = temp;
            }
        }
        return sz * sz;
    }
};







优化 2 vector code:
class Solution {
public:
    int maximalSquare(vector<vector<char>>& matrix) {
        if (matrix.empty()) {
            return 0;
        }
        int m = matrix.size(), n = matrix[0].size(), sz = 0;
        vector<int> pre(n, 0), cur(n, 0);
        for (int i = 0; i < m; i++) {
            for (int j = 0; j < n; j++) {
                if (!i || !j || matrix[i][j] == '0') {
                    cur[j] = matrix[i][j] - '0';
                } else {
                    cur[j] = min(pre[j - 1], min(pre[j], cur[j - 1])) + 1;
                }
                sz = max(cur[j], sz);
            }
            fill(pre.begin(), pre.end(), 0);
            swap(pre, cur);
        }
        return sz * sz;
    }
};
























code:
class Solution {
public:
    int maximalSquare(vector<vector<char>>& matrix) {
        if(matrix.size() == 0) return 0;
        int m = matrix.size(), n = matrix[0].size();
        vector<vector<int>> table(m+1, vector<int>(n+1));
        int res = 0;
        for(int i = 1; i <= m; i++)
        {
            for(int j = 1; j <= n; j++)
            {
                if(matrix[i-1][j-1] == '0') table[i][j] = 0;
                else
                {
                    table[i][j] = min(table[i-1][j-1], min(table[i-1][j], table[i][j-1]))+1;
                    res = max(res, table[i][j]);
                }
            }
        }
        return res*res;
    }
};