给定一个整数数组 prices,其中第 i 个元素代表了第 i 天的股票价格 ;非负整数 fee 代表了交易股票的手续费用。

你可以无限次地完成交易,但是你每次交易都需要付手续费。如果你已经购买了一个股票,在卖出它之前你就不能再继续购买股票了。

返回获得利润的最大值。

示例 1:

输入: prices = [1, 3, 2, 8, 4, 9], fee = 2
输出: 8
解释: 能够达到的最大利润:  
在此处买入 prices[0] = 1
在此处卖出 prices[3] = 8
在此处买入 prices[4] = 4
在此处卖出 prices[5] = 9
总利润: ((8 - 1) - 2) + ((9 - 4) - 2) = 8.
注意:

0 < prices.length <= 50000.
0 < prices[i] < 50000.
0 <= fee < 50000.

























思路:
1.cash 表示当前没有持有股票的利润,hold 表示当前持有股票的利润;
2.股票问题框架解题法。
















框架解题法code:
class Solution {
public:
    int maxProfit(vector<int>& prices, int fee) {
        int dp_i0 = 0, dp_i1 = INT_MIN;
        for(int i = 0; i < prices.size(); i++)
        {
            dp_i0 = max(dp_i0, dp_i1+prices[i]);
            dp_i1 = max(dp_i1, dp_i0-prices[i]-fee);
        }
        return dp_i0;
    }
};





















code:
class Solution {
public:
    int maxProfit(vector<int>& prices, int fee) {
        int cash = 0, hold = -prices[0];
        for(int i = 1; i < prices.size(); i++)
        {
            cash = max(cash, hold+prices[i]-fee);
            hold = max(hold, cash-prices[i]);
        }
        return cash;
    }
};