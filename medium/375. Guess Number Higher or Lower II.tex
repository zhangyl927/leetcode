我们正在玩一个猜数游戏,游戏规则如下:

我从 1 到 n 之间选择一个数字,你来猜我选了哪个数字。

每次你猜错了,我都会告诉你,我选的数字比你的大了或者小了。

然而,当你猜了数字 x 并且猜错了的时候,你需要支付金额为 x 的现金。直到你猜到我选的数字,你才算赢得了这个游戏。

示例:

n = 10, 我选择了8.

第一轮: 你猜我选择的数字是5,我会告诉你,我的数字更大一些,然后你需要支付5块。
第二轮: 你猜是7,我告诉你,我的数字更大一些,你支付7块。
第三轮: 你猜是9,我告诉你,我的数字更小一些,你支付9块。

游戏结束。8 就是我选的数字。

你最终要支付 5 + 7 + 9 = 21 块钱。
给定 n ≥ 1,计算你至少需要拥有多少现金才能确保你能赢得这个游戏。






















思路:
动态规划的方法;
1 2 3 4 5 6 7 8
取 1: 1 + [2 3 4 5]
取 2: 2 + max([1], [3,4,5])
取 3: 3 + max([1,2], [4,5]) 
取 4: 4 + max([1,2,3], [5])
取 5: 5 + [1,2,3,4]





















code:
class Solution {
public:
    int helper(vector<vector<int>>& table, int s, int e)
    {
        if(s >= e) return 0;
        if(table[s][e] != 0) return table[s][e];
        int res = INT_MAX;
        for(int x = s; x <= e; x++)
        {
            int temp = x+max(helper(table, s, x-1), helper(table, x+1, e));
            res = min(res, temp);
        }
        table[s][e] = res;
        return table[s][e];
    }
    int getMoneyAmount(int n) {
        vector<vector<int>> table(n+1, vector<int>(n+1));
        return helper(table, 1, n);
    }
};