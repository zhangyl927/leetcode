一个机器人位于一个 m x n 网格的左上角 (起始点在下图中标记为“Start” )。

机器人每次只能向下或者向右移动一步。机器人试图达到网格的右下角(在下图中标记为“Finish”)。

现在考虑网格中有障碍物。那么从左上角到右下角将会有多少条不同的路径?

网格中的障碍物和空位置分别用 1 和 0 来表示。

说明:m 和 n 的值均不超过 100。

示例 1:

输入:
[
  [0,0,0],
  [0,1,0],
  [0,0,0]
]
输出: 2
解释:
3x3 网格的正中间有一个障碍物。
从左上角到右下角一共有 2 条不同的路径:
1. 向右 -> 向右 -> 向下 -> 向下
2. 向下 -> 向下 -> 向右 -> 向右



























思路:
在 62 题基础上进行修改。
创建 dp 容器存放动态规划过程中的结果。
init 时,若遇到 obstacleGrid[0][i] == 1 时, break;
遍历时,若 遇到障碍物,则continue; 否则 dp[i][j] = dp[i-1][j] + dp[i][j-1];
注意溢出问题。
参考。 https://www.youtube.com/watch?v=O9GhDaafmmo






























code:
class Solution {
public:
    int uniquePathsWithObstacles(vector<vector<int>>& obstacleGrid) {
        int m = obstacleGrid.size(), n = obstacleGrid[0].size();
        vector<vector<int>> dp(m, vector<int>(n));
        for(int i = 0; i < n; i++)
        {
            if(obstacleGrid[0][i] == 1) break;
            dp[0][i] = 1;
        }
        for(int j = 0; j < m; j++)
        {
            if(obstacleGrid[j][0] == 1) break;
            dp[j][0] = 1;
        }
        for(int i = 1; i < m; i++)
        {
            for(int j = 1; j < n; j++)
            {
                if(obstacleGrid[i][j] == 1) continue;
                dp[i][j] = dp[i-1][j] + dp[i][j-1];
            }
        }
        return dp[m-1][n-1];
    }
};