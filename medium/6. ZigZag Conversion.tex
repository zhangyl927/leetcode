将一个给定字符串根据给定的行数,以从上往下、从左到右进行 Z 字形排列。

比如输入字符串为 "LEETCODEISHIRING" 行数为 3 时,排列如下:

L   C   I   R
E T O E S I I G
E   D   H   N

之后,你的输出需要从左往右逐行读取,产生出一个新的字符串,比如:"LCIRETOESIIGEDHN"。

请你实现这个将字符串进行指定行数变换的函数:

string convert(string s, int numRows);

示例 1:

输入: s = "LEETCODEISHIRING", numRows = 3
输出: "LCIRETOESIIGEDHN"

示例 2:

输入: s = "LEETCODEISHIRING", numRows = 4
输出: "LDREOEIIECIHNTSG"
解释:

L     D     R
E   O E   I I
E C   I H   N
T     S     G































思路:
s = "PAYPALISHIRING", numRows = 4
vector<string> res(numRows, "");

P             I           N
  A         L   S       I   G
    Y     A       H   R
       P            I

设置一个 row 和 step,
1. 当 row == 0时, step = 1,此后按 下楼梯 的方向继续, row += step,直到 row == numRows - 1,转到2;
2. 当 row == numRows - 1 时,step = -1,此后按 上楼梯 的方向继续, row += step,直到 row == 0,回到1。
整个过程 呈 ‘W’状。





























code:
class Solution {
public:
    string convert(string s, int numRows) {
        if(numRows <= 1) return s;
        vector<string> res(numRows, "");
        int row = 0, step = 1;
        for(int i = 0; i < s.size(); i++)
        {
            res[row] += s[i];
            if(row == 0) step = 1;
            else if(row == numRows-1) step = -1;
            row += step;
        }
        string rs;
        for(auto a:res)
            rs += a;
        return rs;
    }
};