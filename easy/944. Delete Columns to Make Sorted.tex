给定由 N 个小写字母字符串组成的数组 A,其中每个字符串长度相等。

选取一个删除索引序列,对于 A 中的每个字符串,删除对应每个索引处的字符。 所余下的字符串行从上往下读形成列。

比如,有 A = ["abcdef", "uvwxyz"],删除索引序列 {0, 2, 3},删除后 A 为["bef", "vyz"], A 的列分别为["b","v"], ["e","y"], ["f","z"]。(形式上,第 n 列为 [A[0][n], A[1][n], ..., A[A.length-1][n]])。

假设,我们选择了一组删除索引 D,那么在执行删除操作之后,A 中所剩余的每一列都必须是 非降序 排列的,然后请你返回 D.length 的最小可能值。

示例 1:

输入:["cba", "daf", "ghi"]
输出:1
解释:
当选择 D = {1},删除后 A 的列为:["c","d","g"] 和 ["a","f","i"],均为非降序排列。
若选择 D = {},那么 A 的列 ["b","a","h"] 就不是非降序排列了。

示例 2:

输入:["a", "b"]
输出:0
解释:D = {}

示例 3:

输入:["zyx", "wvu", "tsr"]
输出:3
解释:D = {0, 1, 2}

提示:

    1 <= A.length <= 100
    1 <= A[i].length <= 1000





























思路:
比较每个字符串的同一个位置,不满足非降序的则删除。


























code:
class Solution {
public:
    int minDeletionSize(vector<string>& A) {
        int num = 0;
        for(int i = 0; i < A[0].size(); i++)
        {
            for(int j = 1; j < A.size(); j++)
            {
                if(A[j][i] < A[j-1][i])
                {
                    num++;
                    break;
                }
            }
        }
        return num;
    }
};