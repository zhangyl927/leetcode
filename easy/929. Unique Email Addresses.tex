每封电子邮件都由一个本地名称和一个域名组成,以 @ 符号分隔。

例如,在 alice@leetcode.com中, alice 是本地名称,而 leetcode.com 是域名。

除了小写字母,这些电子邮件还可能包含 '.' 或 '+'。

如果在电子邮件地址的本地名称部分中的某些字符之间添加句点('.'),则发往那里的邮件将会转发到本地名称中没有点的同一地址。例如,"alice.z@leetcode.com” 和 “alicez@leetcode.com” 会转发到同一电子邮件地址。 (请注意,此规则不适用于域名。)

如果在本地名称中添加加号('+'),则会忽略第一个加号后面的所有内容。这允许过滤某些电子邮件,例如 m.y+name@email.com 将转发到 my@email.com。 (同样,此规则不适用于域名。)

可以同时使用这两个规则。

给定电子邮件列表 emails,我们会向列表中的每个地址发送一封电子邮件。实际收到邮件的不同地址有多少?

 

示例:

输入:["test.email+alex@leetcode.com","test.e.mail+bob.cathy@leetcode.com","testemail+david@lee.tcode.com"]
输出:2
解释:实际收到邮件的是 "testemail@leetcode.com" 和 "testemail@lee.tcode.com"。

 

提示:

    1 <= emails[i].length <= 100
    1 <= emails.length <= 100
    每封 emails[i] 都包含有且仅有一个 '@' 字符。






















思路:
直接法。
























code:
class Solution {
public:
    int numUniqueEmails(vector<string>& emails) {
        
        for(int i = 0; i < emails.size(); i++)
        {
            int left = 0, right = 0, point = 0;
            for(int j = 0; emails[i][j] != '@'; j++)
            {
                if(emails[i][j] == '.')
                {
                    emails[i].erase(j,1); j -= 1;
                }
                else if(emails[i][j] == '+' && right == 0)
                {
                    left = j; right = left+1;
                }
                else if(right != 0) right++;
            }
            //return left;
            emails[i].erase(left, right-left);
        }
        unordered_map<string, int> myMap;
        for(auto k:emails)
            myMap[k]++;
        
        int number = 0;
        for(auto m:myMap)
        {
            if(m.second > 0) number++;
        } 
        return number;
    }
};