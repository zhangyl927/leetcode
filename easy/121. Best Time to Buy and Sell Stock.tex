给定一个数组,它的第 i 个元素是一支给定股票第 i 天的价格。

如果你最多只允许完成一笔交易(即买入和卖出一支股票),设计一个算法来计算你所能获取的最大利润。

注意你不能在买入股票前卖出股票。

示例 1:

输入: [7,1,5,3,6,4]
输出: 5
解释: 在第 2 天(股票价格 = 1)的时候买入,在第 5 天(股票价格 = 6)的时候卖出,最大利润 = 6-1 = 5 。
     注意利润不能是 7-1 = 6, 因为卖出价格需要大于买入价格。

示例 2:

输入: [7,6,4,3,1]
输出: 0
解释: 在这种情况下, 没有交易完成, 所以最大利润为 0。





















思路:
1. 动态规划算法。
2. 股票系列框架解法。

























框架解法code:
class Solution {
public:
    int maxProfit(vector<int>& prices) {
        int dp_i_0 = 0, dp_i_1 = INT_MIN;
        for(int i = 0; i < prices.size(); i++)
        {
            dp_i_0 = max(dp_i_0, dp_i_1+prices[i]);
            dp_i_1 = max(dp_i_1, -prices[i]);
        }
        return dp_i_0;
    }
};






















code:
class Solution {
public:
    int maxProfit(vector<int>& prices) {
        if(prices.size() == 0) return 0;
        int sum = 0;
        int max = 0;
        for(int i = 0; i < prices.size()-1; i++)
        {
            sum += prices[i+1]-prices[i];
            if(sum > max) max = sum;
            if(sum < 0) sum = 0;
        }
        return max;
    }
};