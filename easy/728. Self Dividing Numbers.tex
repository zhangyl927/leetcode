自除数 是指可以被它包含的每一位数除尽的数。

例如,128 是一个自除数,因为 128 % 1 == 0,128 % 2 == 0,128 % 8 == 0。

还有,自除数不允许包含 0 。

给定上边界和下边界数字,输出一个列表,列表的元素是边界(含边界)内所有的自除数。

示例 1:

输入: 
上边界left = 1, 下边界right = 22
输出: [1, 2, 3, 4, 5, 6, 7, 8, 9, 11, 12, 15, 22]

注意:

    每个输入参数的边界满足 1 <= left <= right <= 10000。























思路:
直接法,判断每一个数是否为自除数。






















code:
class Solution {
public:
    vector<int> selfDividingNumbers(int left, int right) {
        vector<int> res;
        for(int i = left; i <= right; i++)
        {
            if(selfDividHelper(i)) res.push_back(i);
        }
        return res;
    }
    bool selfDividHelper(int number)
    {
        int origin = number;
        while(number)
        {
            if(number % 10 == 0) return false;
            int temp = number % 10;
            if(origin % temp != 0) return false;
            number /= 10;
        }
        return true; 
    }
};