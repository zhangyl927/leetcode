给定两个整数 L 和 R ,找到闭区间 [L, R] 范围内,计算置位位数为质数的整数个数。

(注意,计算置位代表二进制表示中1的个数。例如 21 的二进制表示 10101 有 3 个计算置位。还有,1 不是质数。)

示例 1:

输入: L = 6, R = 10
输出: 4
解释:
6 -> 110 (2 个计算置位,2 是质数)
7 -> 111 (3 个计算置位,3 是质数)
9 -> 1001 (2 个计算置位,2 是质数)
10-> 1010 (2 个计算置位,2 是质数)

示例 2:

输入: L = 10, R = 15
输出: 5
解释:
10 -> 1010 (2 个计算置位, 2 是质数)
11 -> 1011 (3 个计算置位, 3 是质数)
12 -> 1100 (2 个计算置位, 2 是质数)
13 -> 1101 (3 个计算置位, 3 是质数)
14 -> 1110 (3 个计算置位, 3 是质数)
15 -> 1111 (4 个计算置位, 4 不是质数)

注意:

    L, R 是 L <= R 且在 [1, 10^6] 中的整数。
    R - L 的最大值为 10000。


































思路:
遍历从 L -> R 的数,判断这个数的二进制中 1 的个数是否为质数,若是则 + 1。
构造一个函数来判断这个数的二进制中 1 的个数是否为质数。
































code:
class Solution {
public:
    int countPrimeSetBits(int L, int R) {
        int count = 0;
        for(int i = L; i <= R; i++)
        {
            if(countPrimeSetBitsHelper(i)) count++;
        }
        return count;
    }
    bool countPrimeSetBitsHelper(int x)
    {
        int oneNumber = 0;
        while(x)
        {
            if(x % 2 == 1) oneNumber++;
            x >>= 1;
        }
        if(oneNumber == 1) return false;
        for(int i = 2; i <= sqrt(oneNumber); i++)
        {
            if(oneNumber % i == 0) return false;
        }
        return true;
    }
};