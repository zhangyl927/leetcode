给定由若干 0 和 1 组成的数组 A。我们定义 N_i:从 A[0] 到 A[i] 的第 i 个子数组被解释为一个二进制数(从最高有效位到最低有效位)。

返回布尔值列表 answer,只有当 N_i 可以被 5 整除时,答案 answer[i] 为 true,否则为 false。

 

示例 1:

输入:[0,1,1]
输出:[true,false,false]
解释:
输入数字为 0, 01, 011;也就是十进制中的 0, 1, 3 。只有第一个数可以被 5 整除,因此 answer[0] 为真。

示例 2:

输入:[1,1,1]
输出:[false,false,false]

示例 3:

输入:[0,1,1,1,1,1]
输出:[true,false,false,false,true,false]

示例 4:

输入:[1,1,1,0,1]
输出:[false,false,false,false,false]

 

提示:

    1 <= A.length <= 30000
    A[i] 为 0 或 1

































思路:
直接法会溢出;
求从第一位到当前位的数:number*2 + A[i];可对 10 取余数,如  number == 15 时,取余后为 5,不影响之后的结果。
































code:
class Solution {
public:
    vector<bool> prefixesDivBy5(vector<int>& A) {
        vector<bool> res;
        int number = 0;
        for(auto i:A)
        {
            number = (number*2 + i) % 10;
            res.push_back(number % 5 == 0);
        }
        return res;
    }
};