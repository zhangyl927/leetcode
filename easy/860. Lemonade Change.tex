在柠檬水摊上,每一杯柠檬水的售价为 5 美元。

顾客排队购买你的产品,(按账单 bills 支付的顺序)一次购买一杯。

每位顾客只买一杯柠檬水,然后向你付 5 美元、10 美元或 20 美元。你必须给每个顾客正确找零,也就是说净交易是每位顾客向你支付 5 美元。

注意,一开始你手头没有任何零钱。

如果你能给每位顾客正确找零,返回 true ,否则返回 false 。

示例 1:

输入:[5,5,5,10,20]
输出:true
解释:
前 3 位顾客那里,我们按顺序收取 3 张 5 美元的钞票。
第 4 位顾客那里,我们收取一张 10 美元的钞票,并返还 5 美元。
第 5 位顾客那里,我们找还一张 10 美元的钞票和一张 5 美元的钞票。
由于所有客户都得到了正确的找零,所以我们输出 true。

示例 2:

输入:[5,5,10]
输出:true

示例 3:

输入:[10,10]
输出:false

示例 4:

输入:[5,5,10,10,20]
输出:false
解释:
前 2 位顾客那里,我们按顺序收取 2 张 5 美元的钞票。
对于接下来的 2 位顾客,我们收取一张 10 美元的钞票,然后返还 5 美元。
对于最后一位顾客,我们无法退回 15 美元,因为我们现在只有两张 10 美元的钞票。
由于不是每位顾客都得到了正确的找零,所以答案是 false。

 

提示:

    0 <= bills.length <= 10000
    bills[i] 不是 5 就是 10 或是 20 
































思路:
统计 $5、$10 的数目。根据零钱数目判断是否符合。

























code:
class Solution {
public:
    bool lemonadeChange(vector<int>& bills) {
        vector<int> temp(3,0);
        for(int i:bills)
        {
            if(i == 5) temp[0]++;
            else if(i == 10)
            {
                if(temp[0] < 1) return false;
                else temp[0]--;
                temp[1]++;
            }
            else
            {
                if(temp[0] < 1 || (i > temp[0]*5 + temp[1]*10 + 5)) return false;
                else
                {
                    if(temp[1] > 0)
                    {
                        temp[1]--; temp[0]--;
                    }
                    else temp[0] -= 3;
                    temp[2]++;
                }
            }
        }
        return true;
    }
};