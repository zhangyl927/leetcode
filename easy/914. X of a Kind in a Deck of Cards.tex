给定一副牌,每张牌上都写着一个整数。

此时,你需要选定一个数字 X,使我们可以将整副牌按下述规则分成 1 组或更多组:

    每组都有 X 张牌。
    组内所有的牌上都写着相同的整数。

仅当你可选的 X >= 2 时返回 true。

 

示例 1:

输入:[1,2,3,4,4,3,2,1]
输出:true
解释:可行的分组是 [1,1],[2,2],[3,3],[4,4]

示例 2:

输入:[1,1,1,2,2,2,3,3]
输出:false
解释:没有满足要求的分组。

示例 3:

输入:[1]
输出:false
解释:没有满足要求的分组。

示例 4:

输入:[1,1]
输出:true
解释:可行的分组是 [1,1]

示例 5:

输入:[1,1,2,2,2,2]
输出:true
解释:可行的分组是 [1,1],[2,2],[2,2]


提示:

    1 <= deck.length <= 10000
    0 <= deck[i] < 10000






















思路:
每个数出现的次数的 最大公约数 > 1,则返回true。






























code:
class Solution {
public:
    bool hasGroupsSizeX(vector<int>& deck) {
        unordered_map<int, int> mymap;
        int res = 0;
        for(int i:deck) mymap[i]++;
        for(auto j:mymap) res = __gcd(j.second, res);
        return res > 1;
    }
};