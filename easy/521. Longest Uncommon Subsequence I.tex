给定两个字符串,你需要从这两个字符串中找出最长的特殊序列。最长特殊序列定义如下:该序列为某字符串独有的最长子序列(即不能是其他字符串的子序列)。

子序列可以通过删去字符串中的某些字符实现,但不能改变剩余字符的相对顺序。空序列为所有字符串的子序列,任何字符串为其自身的子序列。

输入为两个字符串,输出最长特殊序列的长度。如果不存在,则返回 -1。

示例 :

输入: "aba", "cdc"
输出: 3
解析: 最长特殊序列可为 "aba" (或 "cdc")

说明:

    两个字符串长度均小于100。
    字符串中的字符仅含有 'a'~'z'。



















思路:
若 a == b,则没有最长特殊子序列;
若 a != b,则较长的字符串必然不是另外一个字符串的子序列,即返回较长字符串的长度;





















code:
class Solution {
public:
    int findLUSlength(string a, string b) {
        if(a == b) return -1;
        else
        {
            if(a.size() > b.size()) return a.size();
            else return b.size();
        }
    }
};