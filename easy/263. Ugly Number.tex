编写一个程序判断给定的数是否为丑数。

丑数就是只包含质因数 2, 3, 5 的正整数。

示例 1:

输入: 6
输出: true
解释: 6 = 2 × 3
示例 2:

输入: 8
输出: true
解释: 8 = 2 × 2 × 2
示例 3:

输入: 14
输出: false 
解释: 14 不是丑数,因为它包含了另外一个质因数 7。
说明:

1 是丑数。
输入不会超过 32 位有符号整数的范围: [−231,  231 − 1]。




















思路:
若 num % 2 != 0 && num % 3 != 0 && num % 5 != 0, 不是丑数;
将 num / 2 或 num / 3 或  num / 5, 最终结果与 1 比较;
























code:
class Solution {
public:
    bool isUgly(int num) {
        if(num == 0) return false;
        if(num == 1) return true;
        if(num % 2 != 0 && num % 3 != 0 && num % 5 != 0)
            return false;
        
        while(num % 2 == 0) num /= 2;
        while(num % 3 == 0) num /= 3;
        while(num % 5 == 0) num /= 5;
        
        return num == 1;
    }
};