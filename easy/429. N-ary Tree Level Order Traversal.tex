给定一个 N 叉树,返回其节点值的层序遍历。 (即从左到右,逐层遍历)。

例如,给定一个 3叉树 :

 

 

返回其层序遍历:

[
     [1],
     [3,2,4],
     [5,6]
]

 

说明:

    树的深度不会超过 1000。
    树的节点总数不会超过 5000。



































思路:
根据二叉树的层序遍历,将节点的左右子树遍历修改为所有子节点的遍历。





















code:
/*
// Definition for a Node.
class Node {
public:
    int val;
    vector<Node*> children;

    Node() {}

    Node(int _val, vector<Node*> _children) {
        val = _val;
        children = _children;
    }
};
*/
class Solution {
public:
    vector<vector<int>> levelOrder(Node* root) {
        vector<vector<int>> res;
        levelorder(root, 0, res);
        return res;
    }
    void levelorder(Node* Node, int depth, vector<vector<int>>& res)
    {
        if(!Node) return;
        if(res.size() == depth)
            res.resize(depth+1);
        
        res[depth].push_back(Node->val);
        for(auto node:Node->children)
            levelorder(node, depth+1, res);
    }
};