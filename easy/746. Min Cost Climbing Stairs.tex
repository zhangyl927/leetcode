数组的每个索引做为一个阶梯,第 i个阶梯对应着一个非负数的体力花费值 cost[i](索引从0开始)。

每当你爬上一个阶梯你都要花费对应的体力花费值,然后你可以选择继续爬一个阶梯或者爬两个阶梯。

您需要找到达到楼层顶部的最低花费。在开始时,你可以选择从索引为 0 或 1 的元素作为初始阶梯。

示例 1:

输入: cost = [10, 15, 20]
输出: 15
解释: 最低花费是从cost[1]开始,然后走两步即可到阶梯顶,一共花费15。

 示例 2:

输入: cost = [1, 100, 1, 1, 1, 100, 1, 1, 100, 1]
输出: 6
解释: 最低花费方式是从cost[0]开始,逐个经过那些1,跳过cost[3],一共花费6。

注意:

    cost 的长度将会在 [2, 1000]。
    每一个 cost[i] 将会是一个Integer类型,范围为 [0, 999]。






























思路:
动态规划。


























code:
class Solution {
public:
    int minCostClimbingStairs(vector<int>& cost) {
        for(int i = 2; i < cost.size(); i++)
                cost[i] = cost[i] + min(cost[i-2], cost[i-1]);
        return min(cost[cost.size()-1], cost[cost.size()-2]);
    }
};