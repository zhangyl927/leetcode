给定两个没有重复元素的数组 nums1 和 nums2 ,其中nums1 是 nums2 的子集。找到 nums1 中每个元素在 nums2 中的下一个比其大的值。

nums1 中数字 x 的下一个更大元素是指 x 在 nums2 中对应位置的右边的第一个比 x 大的元素。如果不存在,对应位置输出-1。

示例 1:

输入: nums1 = [4,1,2], nums2 = [1,3,4,2].
输出: [-1,3,-1]
解释:
    对于num1中的数字4,你无法在第二个数组中找到下一个更大的数字,因此输出 -1。
    对于num1中的数字1,第二个数组中数字1右边的下一个较大数字是 3。
    对于num1中的数字2,第二个数组中没有下一个更大的数字,因此输出 -1。
示例 2:

输入: nums1 = [2,4], nums2 = [1,2,3,4].
输出: [3,-1]
解释:
    对于num1中的数字2,第二个数组中的下一个较大数字是3。
    对于num1中的数字4,第二个数组中没有下一个更大的数字,因此输出 -1。
注意:

nums1和nums2中所有元素是唯一的。
nums1和nums2 的数组大小都不超过1000。



























思路1:
使用 find() 函数找到 num1 当前数字在 num2 中的位置,在此位置后面找比 num1 当前数字大的数。若找不到则为 -1,否则填入该数字。





思路2:
直接法。




思路3: 
map<value, index>。将 nums2 的 值和索引 存放在 map 中。





思路4:
单调栈。
对 nums2 从最后一位往前扫描
while stack != empty && current value 大于 stack.top 的值时
	stack pop
if(stack.empty()) map[current value] = -1;
else map[current value] = stack.top;
stack.push(current value);

将 num2 中每位数的 val 和 next val 存放在 map 中了,根据 num1 的值在 map 中查找即可得到答案。

























code1:
class Solution {
public:
    vector<int> nextGreaterElement(vector<int>& nums1, vector<int>& nums2) {
        vector<int> res;
        for(int i = 0; i < nums1.size(); i++)
        {
            vector<int>::iterator it = std::find(nums2.begin(), nums2.end(), nums1[i]);
            while(it != nums2.end())
            {
                if(*it > nums1[i])
                {
                    res.push_back(*it);
                    break;
                }
                else it++;
            }
            if(it == nums2.end()) res.push_back(-1);
        }
        return res;
    }
};




code2:
class Solution {
public:
    vector<int> nextGreaterElement(vector<int>& nums1, vector<int>& nums2) {
        vector<int> res;
        for(int i = 0; i < nums1.size(); i++)
        {
            bool flag = false;
            int j = 0;
            for(; j < nums2.size(); j++)
            {
                if(flag && nums2[j] > nums1[i])
                {
                    res.push_back(nums2[j]); break;
                }
                if(nums2[j] == nums1[i]) flag = true;
            }
            if(j == nums2.size()) res.push_back(-1);
        }
        return res;
    }
};





code3:
class Solution {
public:
    vector<int> nextGreaterElement(vector<int>& nums1, vector<int>& nums2) {
        vector<int> res;
        unordered_map<int, int> nums2map;
        for(int p = 0; p < nums2.size(); p++)
            nums2map[nums2[p]] = p;
        
        for(auto c:nums1)
        { 
            int j = nums2map[c];
            for(; j < nums2.size(); j++)
            {
                if(nums2[j] > c)
                {
                    res.push_back(nums2[j]); break;
                }  
            }
            if(j == nums2.size()) res.push_back(-1);
        }
        return res;
    }
};





code4:
class Solution {
public:
    vector<int> nextGreaterElement(vector<int>& nums1, vector<int>& nums2) {
        vector<int> res;
        stack<int> Stack;
        unordered_map<int, int> map2;
        for(int i = nums2.size()-1; i >= 0; i--)
        {
            while(!Stack.empty() && nums2[i] > Stack.top())
                Stack.pop();
            if(Stack.empty()) map2[nums2[i]] = -1;
            else map2[nums2[i]] = Stack.top();
            Stack.push(nums2[i]);
        }
        for(auto c:nums1) res.push_back(map2[c]);
        return res;
    }
};