编写一个程序,找到两个单链表相交的起始节点。

如下面的两个链表:

在节点 c1 开始相交。

 

示例 1:

输入:intersectVal = 8, listA = [4,1,8,4,5], listB = [5,0,1,8,4,5], skipA = 2, skipB = 3
输出:Reference of the node with value = 8
输入解释:相交节点的值为 8 (注意,如果两个列表相交则不能为 0)。从各自的表头开始算起,链表 A 为 [4,1,8,4,5],链表 B 为 [5,0,1,8,4,5]。在 A 中,相交节点前有 2 个节点;在 B 中,相交节点前有 3 个节点。

 

示例 2:

输入:intersectVal = 2, listA = [0,9,1,2,4], listB = [3,2,4], skipA = 3, skipB = 1
输出:Reference of the node with value = 2
输入解释:相交节点的值为 2 (注意,如果两个列表相交则不能为 0)。从各自的表头开始算起,链表 A 为 [0,9,1,2,4],链表 B 为 [3,2,4]。在 A 中,相交节点前有 3 个节点;在 B 中,相交节点前有 1 个节点。

 

示例 3:

输入:intersectVal = 0, listA = [2,6,4], listB = [1,5], skipA = 3, skipB = 2
输出:null
输入解释:从各自的表头开始算起,链表 A 为 [2,6,4],链表 B 为 [1,5]。由于这两个链表不相交,所以 intersectVal 必须为 0,而 skipA 和 skipB 可以是任意值。
解释:这两个链表不相交,因此返回 null。

 

注意:

    如果两个链表没有交点,返回 null.
    在返回结果后,两个链表仍须保持原有的结构。
    可假定整个链表结构中没有循环。
    程序尽量满足 O(n) 时间复杂度,且仅用 O(1) 内存。





























思路:
!lA 时,lA = headB;!lB时,lB = headA;仅当 lA 和 lB 均为 NULL 时,才表明无交点。




























code:
/**
 * Definition for singly-linked list.
 * struct ListNode {
 *     int val;
 *     ListNode *next;
 *     ListNode(int x) : val(x), next(NULL) {}
 * };
 */
class Solution {
public:
    ListNode *getIntersectionNode(ListNode *headA, ListNode *headB) {
        ListNode* lA = headA;
        ListNode* lB = headB;
        if(!lA || !lB) return NULL;
        
        while(lA != lB)
        {
            lA = (lA == NULL) ? headB : lA->next;
            lB = (lB == NULL) ? headA : lB->next;
        }
        return lA;
    }
};