给定一个 n 个元素有序的(升序)整型数组 nums 和一个目标值 target  ,写一个函数搜索 nums 中的 target,如果目标值存在返回下标,否则返回 -1。


示例 1:

输入: nums = [-1,0,3,5,9,12], target = 9
输出: 4
解释: 9 出现在 nums 中并且下标为 4

示例 2:

输入: nums = [-1,0,3,5,9,12], target = 2
输出: -1
解释: 2 不存在 nums 中因此返回 -1

 

提示:

    你可以假设 nums 中的所有元素是不重复的。
    n 将在 [1, 10000]之间。
    nums 的每个元素都将在 [-9999, 9999]之间。




























思路:
二分法。

























code:
class Solution {
public:
    int search(vector<int>& nums, int target) {
        if(nums[0] > target) return -1;
        return search_number(nums, 0, nums.size()-1, target);
    }
    int search_number(vector<int>& nums, int begin, int end, int target)
    {
        if(begin == end)
        {
            if(nums[begin] == target) return begin;
            else return -1;
        }
        int middle = begin + (end-begin)/2;
        if(nums[middle] == target) return middle;
        else if(nums[middle] < target) return search_number(nums, middle+1, end, target);
        else return search_number(nums, begin, middle-1, target);
    }
};