给定一个非空数组,返回此数组中第三大的数。如果不存在,则返回数组中最大的数。要求算法时间复杂度必须是O(n)。

示例 1:

输入: [3, 2, 1]

输出: 1

解释: 第三大的数是 1.
示例 2:

输入: [1, 2]

输出: 2

解释: 第三大的数不存在, 所以返回最大的数 2 .
示例 3:

输入: [2, 2, 3, 1]

输出: 1

解释: 注意,要求返回第三大的数,是指第三大且唯一出现的数。
存在两个值为2的数,它们都排第二。





















思路:
排序,将数组中重复的部分去掉。


























code:
class Solution {
public:
    int thirdMax(vector<int>& nums) {
        // 排序
        sort(nums.begin(), nums.end());
        vector<int>::iterator it = nums.begin()+1;
        // 将数组中重复的部分去掉
        while(it != nums.end())
        {
            if(*it == *(it-1))
                it = nums.erase(it);
            else it++;
        }
        if(nums.size() <= 2) return nums[nums.size()-1];
        else return nums[nums.size()-3];
    }
};