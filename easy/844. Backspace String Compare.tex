给定 S 和 T 两个字符串,当它们分别被输入到空白的文本编辑器后,判断二者是否相等,并返回结果。 # 代表退格字符。

 

示例 1:

输入:S = "ab#c", T = "ad#c"
输出:true
解释:S 和 T 都会变成 “ac”。
示例 2:

输入:S = "ab##", T = "c#d#"
输出:true
解释:S 和 T 都会变成 “”。
示例 3:

输入:S = "a##c", T = "#a#c"
输出:true
解释:S 和 T 都会变成 “c”。
示例 4:

输入:S = "a#c", T = "b"
输出:false
解释:S 会变成 “c”,但 T 仍然是 “b”。
 

提示:

1 <= S.length <= 200
1 <= T.length <= 200
S 和 T 只含有小写字母以及字符 '#'。



























思路:
将S 和 T 放入两个堆栈中,比较两个堆栈中的元素是否相同。注意堆栈为空时放入 “#” 的情况。



























code:
class Solution {
public:
    bool backspaceCompare(string S, string T) {
        stack<int> S_stack;
        stack<int> T_stack;
        for(int i = 0; i < S.length(); i++)
        {
            if(S[i] == '#')
            {
                if(!S_stack.empty()) S_stack.pop();
                else continue;
            }       
            else S_stack.push(S[i]);
        }
        for(int j = 0; j < T.length(); j++)
        {
            if(T[j] == '#')
            {
                if(!T_stack.empty()) T_stack.pop();
                else continue;
            }  
            else T_stack.push(T[j]);
        }
        if(S_stack.size() != T_stack.size()) return false;
        while(!S_stack.empty() && !T_stack.empty())
        {
            if(S_stack.top() != T_stack.top())
                return false;
            S_stack.pop();
            T_stack.pop();
        }
        if(!S_stack.empty() || !T_stack.empty())
            return false;
        return true;
    }
};