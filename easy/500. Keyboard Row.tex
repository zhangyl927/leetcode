给定一个单词列表,只返回可以使用在键盘同一行的字母打印出来的单词。键盘如下图所示。

 

American keyboard

 

示例:

输入: ["Hello", "Alaska", "Dad", "Peace"]
输出: ["Alaska", "Dad"]

 

注意:

    你可以重复使用键盘上同一字符。
    你可以假设输入的字符串将只包含字母。

















思路:
创建一个 map<char, int>, 每个字符及其行号放进 map 中。
遍历容器中的字符串,将字符串中的的字符全部转换为小写字母,取出字符串中每个字符的 行号进行比较,全部相同则将该字符串放入结果中。
















code:
class Solution {
public:
    vector<string> findWords(vector<string>& words) {
        vector<string> s;
        unordered_map<char, int> line;
        line['q'] = 1;
        line['w'] = 1;
        line['e'] = 1;
        line['r'] = 1;
        line['t'] = 1;
        line['y'] = 1;
        line['u'] = 1;
        line['i'] = 1;
        line['o'] = 1;
        line['p'] = 1;
        line['a'] = 2;
        line['s'] = 2;
        line['d'] = 2;
        line['f'] = 2;
        line['g'] = 2;
        line['h'] = 2;
        line['j'] = 2;
        line['k'] = 2;
        line['l'] = 2;
        line['z'] = 3;
        line['x'] = 3;
        line['c'] = 3;
        line['v'] = 3;
        line['b'] = 3;
        line['n'] = 3;
        line['m'] = 3;
        
        for(int i = 0; i < words.size(); i++)
        {
            string temp = words[i];
            for(auto &c:words[i])
            {
                if(c >= 'A' && c <= 'Z') c += 32;   
            }
            int flag = 0;
            for(int j = 1; flag == 0 && j < words[i].size(); j++)
            {
                if(line[words[i][j]] != line[words[i][j-1]])
                {
                    flag = 1; j = words[i].size();
                }
            }
            words[i] = temp;
            if(flag == 0) s.push_back(words[i]);
        }
        return s;
    }
};