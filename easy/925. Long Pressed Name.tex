你的朋友正在使用键盘输入他的名字 name。偶尔,在键入字符 c 时,按键可能会被长按,而字符可能被输入 1 次或多次。

你将会检查键盘输入的字符 typed。如果它对应的可能是你的朋友的名字(其中一些字符可能被长按),那么就返回 True。

 

示例 1:

输入:name = "alex", typed = "aaleex"
输出:true
解释:'alex' 中的 'a' 和 'e' 被长按。

示例 2:

输入:name = "saeed", typed = "ssaaedd"
输出:false
解释:'e' 一定需要被键入两次,但在 typed 的输出中不是这样。

示例 3:

输入:name = "leelee", typed = "lleeelee"
输出:true

示例 4:

输入:name = "laiden", typed = "laiden"
输出:true
解释:长按名字中的字符并不是必要的。

 

提示:

    name.length <= 1000
    typed.length <= 1000
    name 和 typed 的字符都是小写字母。

















思路:
name 与 typed 中逐位比较,若不相同,则看 typed 中是否是重复,若不是重复则返回 false,若是重复,则 typed 往前移一位 继续与 name 比较。



















code:
class Solution {
public:
    int isLongPressedName(string name, string typed) {
        if(name[0] != typed[0] || name.size() > typed.size()) return false;
        int i = 1, j = 1;
        while(i < name.size() && j < typed.size())
        {
            int front_char = typed[j-1];
            if(name[i] != typed[j])
            {
                if(typed[j] == front_char) j++;
                else return false;
            }
            else
            {
                i++;j++;
            }
        }
        if(i < name.size()) return false;
        while(j < typed.size())
        {
            if(typed[j] != name[name.size()-1]) return false;
            j++;
        }
        return true;
    }
};