如果一个矩阵的每一方向由左上到右下的对角线上具有相同元素,那么这个矩阵是托普利茨矩阵。

给定一个 M x N 的矩阵,当且仅当它是托普利茨矩阵时返回 True。

示例 1:

输入: 
matrix = [
  [1,2,3,4],
  [5,1,2,3],
  [9,5,1,2]
]
输出: True
解释:
在上述矩阵中, 其对角线为:
"[9]", "[5, 5]", "[1, 1, 1]", "[2, 2, 2]", "[3, 3]", "[4]"。
各条对角线上的所有元素均相同, 因此答案是True。

示例 2:

输入:
matrix = [
  [1,2],
  [2,2]
]
输出: False
解释: 
对角线"[1, 2]"上的元素不同。

说明:

     matrix 是一个包含整数的二维数组。
    matrix 的行数和列数均在 [1, 20]范围内。
    matrix[i][j] 包含的整数在 [0, 99]范围内。

进阶:

    如果矩阵存储在磁盘上,并且磁盘内存是有限的,因此一次最多只能将一行矩阵加载到内存中,该怎么办?
    如果矩阵太大以至于只能一次将部分行加载到内存中,该怎么办?


























思路:
比较每条对角线上的数是否相等。






























code:
class Solution {
public:
    bool isToeplitzMatrix(vector<vector<int>>& matrix) {
        int r = matrix.size();
        int c = matrix[0].size();
        
        for(int i = 0; i < r-1; i++)
            for(int j = 0; j < c - 1; j++)
            {
                if(matrix[i][j] != matrix[i+1][j+1]) return false;
            }
        
        return true;
    }
};