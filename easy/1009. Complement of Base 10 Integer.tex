每个非负整数 N 都有其二进制表示。例如, 5 可以被表示为二进制 "101",11 可以用二进制 "1011" 表示,依此类推。注意,除 N = 0 外,任何二进制表示中都不含前导零。

二进制的反码表示是将每个 1 改为 0 且每个 0 变为 1。例如,二进制数 "101" 的二进制反码为 "010"。

给定十进制数 N,返回其二进制表示的反码所对应的十进制整数。

 

示例 1:

输入:5
输出:2
解释:5 的二进制表示为 "101",其二进制反码为 "010",也就是十进制中的 2 。

示例 2:

输入:7
输出:0
解释:7 的二进制表示为 "111",其二进制反码为 "000",也就是十进制中的 0 。

示例 3:

输入:10
输出:5
解释:10 的二进制表示为 "1010",其二进制反码为 "0101",也就是十进制中的 5 。

 

提示:

    0 <= N < 10^9

































思路:
直接法。


































code:
class Solution {
public:
    int bitwiseComplement(int N) {
        vector<int> res;
        if(N == 0) return 1;
        while(N)
        {
            if(N % 2 == 1) res.push_back(0);
            else res.push_back(1);
            N >>= 1;
        }
        reverse(res.begin(), res.end());
        int result = 0;
        for(auto c:res)  result = result*2 + c;
        return result;
    }
};