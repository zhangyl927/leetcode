给定一个数组 nums 和一个值 val,你需要原地移除所有数值等于 val 的元素,返回移除后数组的新长度。

不要使用额外的数组空间,你必须在原地修改输入数组并在使用 O(1) 额外空间的条件下完成。

元素的顺序可以改变。你不需要考虑数组中超出新长度后面的元素。

示例 1:

给定 nums = [3,2,2,3], val = 3,

函数应该返回新的长度 2, 并且 nums 中的前两个元素均为 2。

你不需要考虑数组中超出新长度后面的元素。
示例 2:

给定 nums = [0,1,2,2,3,0,4,2], val = 2,

函数应该返回新的长度 5, 并且 nums 中的前五个元素为 0, 1, 3, 0, 4。

注意这五个元素可为任意顺序。

你不需要考虑数组中超出新长度后面的元素。
说明:

为什么返回数值是整数,但输出的答案是数组呢?

请注意,输入数组是以“引用”方式传递的,这意味着在函数里修改输入数组对于调用者是可见的。

你可以想象内部操作如下:

// nums 是以“引用”方式传递的。也就是说,不对实参作任何拷贝
int len = removeElement(nums, val);

// 在函数里修改输入数组对于调用者是可见的。
// 根据你的函数返回的长度, 它会打印出数组中该长度范围内的所有元素。
for (int i = 0; i < len; i++) {
    print(nums[i]);
}


























思路:
使用 erase 方法。



























code:
class Solution {
public:
    int removeElement(vector<int>& nums, int val) {
        if(nums.size() == 0) return 0;
        
        vector<int>::iterator iterator;
        iterator = nums.begin();
        while(iterator != nums.end())
        {
            if(*iterator == val)
                iterator = nums.erase(iterator);
            else iterator++;
        }
        return nums.size();
    }
};