给定一个由空格分割单词的句子 S。每个单词只包含大写或小写字母。

我们要将句子转换为 “Goat Latin”(一种类似于 猪拉丁文 - Pig Latin 的虚构语言)。

山羊拉丁文的规则如下:

    如果单词以元音开头(a, e, i, o, u),在单词后添加"ma"。
    例如,单词"apple"变为"applema"。

    如果单词以辅音字母开头(即非元音字母),移除第一个字符并将它放到末尾,之后再添加"ma"。
    例如,单词"goat"变为"oatgma"。

    根据单词在句子中的索引,在单词最后添加与索引相同数量的字母'a',索引从1开始。
    例如,在第一个单词后添加"a",在第二个单词后添加"aa",以此类推。

返回将 S 转换为山羊拉丁文后的句子。

示例 1:

输入: "I speak Goat Latin"
输出: "Imaa peaksmaaa oatGmaaaa atinLmaaaaa"

示例 2:

输入: "The quick brown fox jumped over the lazy dog"
输出: "heTmaa uickqmaaa rownbmaaaa oxfmaaaaa umpedjmaaaaaa overmaaaaaaa hetmaaaaaaaa azylmaaaaaaaaa ogdmaaaaaaaaaa"

说明:

    S 中仅包含大小写字母和空格。单词间有且仅有一个空格。
    1 <= S.length <= 150。




















思路:
直接法。



















code:
class Solution {
public:
    string toGoatLatin(string S) {
        istringstream in(S);
        string t;
        vector<string> sv;
        while(in >> t)
            sv.push_back(t);
        string res = "";
        for(int j = 0; j < sv.size(); j++)
        {
            if(sv[j][0] == 'a' || sv[j][0] == 'e' || sv[j][0] == 'i' || sv[j][0] == 'o' ||sv[j][0] == 'u' || sv[j][0] == 'A' || sv[j][0] == 'E' || sv[j][0] == 'I' || sv[j][0] == 'O' ||sv[j][0] == 'U')
            {
                res += sv[j] + "ma";
            }
            else
            {
                reverse(sv[j].begin(), sv[j].end());
                reverse(sv[j].begin(), sv[j].end()-1);
                res += sv[j] + "ma";
            }
            int k = j+1;
            while(k)
            {
                res += 'a'; k--;
            }
            res += " ";
        }
        res.pop_back();
        return res;
    }
};