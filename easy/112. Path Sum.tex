给定一个二叉树和一个目标和,判断该树中是否存在根节点到叶子节点的路径,这条路径上所有节点值相加等于目标和。

说明: 叶子节点是指没有子节点的节点。

示例: 
给定如下二叉树,以及目标和 sum = 22,

              5
             / \
            4   8
           /   / \
          11  13  4
         /  \      \
        7    2      1

返回 true, 因为存在目标和为 22 的根节点到叶子节点的路径 5->4->11->2。

























思路:
遍历从根节点到当前节点的和与 sum 的大小,当当前节点为叶子节点且和等于 sum 时,返回 true;否则继续遍历,若没找到,返回 false。


























/**
 * Definition for a binary tree node.
 * struct TreeNode {
 *     int val;
 *     TreeNode *left;
 *     TreeNode *right;
 *     TreeNode(int x) : val(x), left(NULL), right(NULL) {}
 * };
 */
class Solution {
public:
    bool hasPathSum(TreeNode* root, int sum) {
        if(!root) return false;
        if(!root->left && !root->right) return root->val == sum;
        return hasPathSum(root->left, sum-root->val) || hasPathSum(root->right, sum-root->val);
    }
};