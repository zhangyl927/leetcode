给定两个正整数 x 和 y,如果某一整数等于 x^i + y^j,其中整数 i >= 0 且 j >= 0,那么我们认为该整数是一个强整数。

返回值小于或等于 bound 的所有强整数组成的列表。

你可以按任何顺序返回答案。在你的回答中,每个值最多出现一次。

 

示例 1:

输入:x = 2, y = 3, bound = 10
输出:[2,3,4,5,7,9,10]
解释: 
2 = 2^0 + 3^0
3 = 2^1 + 3^0
4 = 2^0 + 3^1
5 = 2^1 + 3^1
7 = 2^2 + 3^1
9 = 2^3 + 3^0
10 = 2^0 + 3^2

示例 2:

输入:x = 3, y = 5, bound = 15
输出:[2,4,6,8,10,14]

 

提示:

    1 <= x <= 100
    1 <= y <= 100
    0 <= bound <= 10^6































思路:
1. 求出 x 的最大指数 x_max,y 的最大指数 y_max。
2. 求 符合 pow(x, i) + pow(y, j) <= bound 的数。
3. 消除重复项。 






























code:
class Solution {
public:
    vector<int> powerfulIntegers(int x, int y, int bound) {
        if(bound <= 1) return {};
        int x_max = 0, y_max = 0, xflag = 0, yflag = 0;
        vector<int> res;
        for(int i = 0; i < bound; i++)
        {
            if(pow(x, i) <= bound && pow(x, i+1) > bound && xflag == 0)
            {
                x_max = i; xflag = 1;
            }
            if(pow(y, i) <= bound && pow(y, i+1) > bound && yflag == 0)
            {
                y_max = i; yflag = 1;
            }
            if(xflag == 1 && yflag == 1) break;
        }
        for(int m = 0; m <= x_max; m++)
        {
            for(int n = 0; n <= y_max; n++)
            {
                if(pow(x,m) + pow(y,n) <= bound) res.push_back(pow(x,m) + pow(y,n)); 
            }
        }
        sort(res.begin(), res.end());
        vector<int> result(1,res[0]);
        for(int k = 1; k < res.size(); k++)
        {
            if(res[k] != res[k-1]) result.push_back(res[k]);
        }
        return result;
    }
};