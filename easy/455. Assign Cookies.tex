假设你是一位很棒的家长,想要给你的孩子们一些小饼干。但是,每个孩子最多只能给一块饼干。对每个孩子 i ,都有一个胃口值 gi ,这是能让孩子们满足胃口的饼干的最小尺寸;并且每块饼干 j ,都有一个尺寸 sj 。如果 sj >= gi ,我们可以将这个饼干 j 分配给孩子 i ,这个孩子会得到满足。你的目标是尽可能满足越多数量的孩子,并输出这个最大数值。

注意:

你可以假设胃口值为正。
一个小朋友最多只能拥有一块饼干。

示例 1:

输入: [1,2,3], [1,1]

输出: 1

解释: 
你有三个孩子和两块小饼干,3个孩子的胃口值分别是:1,2,3。
虽然你有两块小饼干,由于他们的尺寸都是1,你只能让胃口值是1的孩子满足。
所以你应该输出1。

示例 2:

输入: [1,2], [1,2,3]

输出: 2

解释: 
你有两个孩子和三块小饼干,2个孩子的胃口值分别是1,2。
你拥有的饼干数量和尺寸都足以让所有孩子满足。
所以你应该输出2.

























思路:
将两个数组排序比较。
























code:
class Solution {
public:
    int findContentChildren(vector<int>& g, vector<int>& s) {
        sort(g.begin(), g.end());
        sort(s.begin(), s.end());
        int i = 0, j = 0;
        int number = 0;
        while(i < g.size() && j < s.size())
        {
            if(s[j] >= g[i])
            {
                number++;
                i++; j++;
            }
            else j++;
        }
        return number;
    }
};