颠倒给定的 32 位无符号整数的二进制位。

 

示例 1:

输入: 00000010100101000001111010011100
输出: 00111001011110000010100101000000
解释: 输入的二进制串 00000010100101000001111010011100 表示无符号整数 43261596,
      因此返回 964176192,其二进制表示形式为 00111001011110000010100101000000。
示例 2:

输入:11111111111111111111111111111101
输出:10111111111111111111111111111111
解释:输入的二进制串 11111111111111111111111111111101 表示无符号整数 4294967293,
      因此返回 3221225471 其二进制表示形式为 10101111110010110010011101101001。
 

提示:

请注意,在某些语言(如 Java)中,没有无符号整数类型。在这种情况下,输入和输出都将被指定为有符号整数类型,并且不应影响您的实现,因为无论整数是有符号的还是无符号的,其内部的二进制表示形式都是相同的。
在 Java 中,编译器使用二进制补码记法来表示有符号整数。因此,在上面的 示例 2 中,输入表示有符号整数 -3,输出表示有符号整数 -1073741825。



















思路:
定义一个数等于0,每次左移一位补 0,将这位加上 n & 1 的结果,n >>= 1。




















code:
class Solution {
public:
    uint32_t reverseBits(uint32_t n) {
        uint32_t result = 0;
        int i = 32;
        while(i--)
        {
            result <<= 1;
            result += n&1;
            n >>= 1;
        }
        return result;
    }
};