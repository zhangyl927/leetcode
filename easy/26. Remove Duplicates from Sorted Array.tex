给定一个排序数组,你需要在原地删除重复出现的元素,使得每个元素只出现一次,返回移除后数组的新长度。

不要使用额外的数组空间,你必须在原地修改输入数组并在使用 O(1) 额外空间的条件下完成。

示例 1:

给定数组 nums = [1,1,2], 

函数应该返回新的长度 2, 并且原数组 nums 的前两个元素被修改为 1, 2。 

你不需要考虑数组中超出新长度后面的元素。
示例 2:

给定 nums = [0,0,1,1,1,2,2,3,3,4],

函数应该返回新的长度 5, 并且原数组 nums 的前五个元素被修改为 0, 1, 2, 3, 4。

你不需要考虑数组中超出新长度后面的元素。
说明:

为什么返回数值是整数,但输出的答案是数组呢?

请注意,输入数组是以“引用”方式传递的,这意味着在函数里修改输入数组对于调用者是可见的。

你可以想象内部操作如下:

// nums 是以“引用”方式传递的。也就是说,不对实参做任何拷贝
int len = removeDuplicates(nums);

// 在函数里修改输入数组对于调用者是可见的。
// 根据你的函数返回的长度, 它会打印出数组中该长度范围内的所有元素。
for (int i = 0; i < len; i++) {
    print(nums[i]);
}


























思路:
使用 80 题的思路,设置两个指针 index = 1 和 lock = 1,index 遍历 数组,lock 表明去重后的 数组。

使用迭代器,当前数等于前一个数时,使用 erase 在数组中删除当前的数。






















优化code:
class Solution {
public:
    int removeDuplicates(vector<int>& nums) {
        if(nums.size() == 0) return 0;
        int index = 1, lock = 1;
        for(; index < nums.size(); index++)
        {
            if(!(nums[index] == nums[lock-1]))
                nums[lock++] = nums[index];
        }
        return lock;
    }
};










code:
class Solution {
public:
    int removeDuplicates(vector<int>& nums) {
        if(nums.size() == 0) return 0;
        
        vector<int>::iterator iterator;
        iterator = nums.begin() + 1;
        while(iterator != nums.end())
        {
            if(*iterator == *(iterator-1))
            {
                iterator = nums.erase(iterator);
            }
            else iterator++;
        }
        return nums.size();
    }
};