在二叉树中,根节点位于深度 0 处,每个深度为 k 的节点的子节点位于深度 k+1 处。

如果二叉树的两个节点深度相同,但父节点不同,则它们是一对堂兄弟节点。

我们给出了具有唯一值的二叉树的根节点 root,以及树中两个不同节点的值 x 和 y。

只有与值 x 和 y 对应的节点是堂兄弟节点时,才返回 true。否则,返回 false。

 

示例 1:

输入:root = [1,2,3,4], x = 4, y = 3
输出:false

示例 2:

输入:root = [1,2,3,null,4,null,5], x = 5, y = 4
输出:true

示例 3:

输入:root = [1,2,3,null,4], x = 2, y = 3
输出:false

 

提示:

    二叉树的节点数介于 2 到 100 之间。
    每个节点的值都是唯一的、范围为 1 到 100 的整数。





















思路:
层序遍历将节点放入 vector<vector<int>> res 中,同时找到 x、y 的 父节点 x_Father、y_Father。
若 x、y 在 res 的同一个子容器中,且 x_Father != y_Father,则是堂兄弟节点。





















code:
/**
 * Definition for a binary tree node.
 * struct TreeNode {
 *     int val;
 *     TreeNode *left;
 *     TreeNode *right;
 *     TreeNode(int x) : val(x), left(NULL), right(NULL) {}
 * };
 */
class Solution {
public:
    TreeNode* xFather = NULL;
    TreeNode* yFather = NULL;
    bool isCousins(TreeNode* root, int x, int y) {
        vector<vector<int>> res;
        helper(root, x, y, 0, res);
        int x_depth = 0, y_depth = 0;
        for(int i = 0; i < res.size(); i++)
        {   
            for(int k = 0; k < res[i].size(); k++)
            {
                if(res[i][k] == x)  x_depth = i;
                if(res[i][k] == y)  y_depth = i;
            }
        }
        return (x_depth == y_depth) && (xFather != yFather);
    }
    void helper(TreeNode* Node,int x, int y, int depth, vector<vector<int>>& res)
    {
        if(!Node) return;
        if(Node->left && Node->left->val == x) xFather = Node;
        if(Node->right && Node->right->val == x) xFather = Node;
        if(Node->left && Node->left->val == y) yFather = Node;
        if(Node->right && Node->right->val == y) yFather = Node;
        
        if(res.size() == depth)
            res.resize(depth+1);
        res[depth].push_back(Node->val);
        helper(Node->left, x, y, depth+1, res);
        helper(Node->right, x, y, depth+1, res);
    }
};