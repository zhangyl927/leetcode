给出一个整数数组 A 和一个查询数组 queries。

对于第 i 次查询,有 val = queries[i][0], index = queries[i][1],我们会把 val 加到 A[index] 上。然后,第 i 次查询的答案是 A 中偶数值的和。

(此处给定的 index = queries[i][1] 是从 0 开始的索引,每次查询都会永久修改数组 A。)

返回所有查询的答案。你的答案应当以数组 answer 给出,answer[i] 为第 i 次查询的答案。

 

示例:

输入:A = [1,2,3,4], queries = [[1,0],[-3,1],[-4,0],[2,3]]
输出:[8,6,2,4]
解释:
开始时,数组为 [1,2,3,4]。
将 1 加到 A[0] 上之后,数组为 [2,2,3,4],偶数值之和为 2 + 2 + 4 = 8。
将 -3 加到 A[1] 上之后,数组为 [2,-1,3,4],偶数值之和为 2 + 4 = 6。
将 -4 加到 A[0] 上之后,数组为 [-2,-1,3,4],偶数值之和为 -2 + 4 = 2。
将 2 加到 A[3] 上之后,数组为 [-2,-1,3,6],偶数值之和为 -2 + 6 = 4。

 

提示:

    1 <= A.length <= 10000
    -10000 <= A[i] <= 10000
    1 <= queries.length <= 10000
    -10000 <= queries[i][0] <= 10000
    0 <= queries[i][1] < A.length























思路:
先计算数组中的偶数之和even_sum。
每进行一次查询:1.判断 even_sum 是否发生变化。   2.对数组中的数进行修改。

























code:
class Solution {
public:
    vector<int> sumEvenAfterQueries(vector<int>& A, vector<vector<int>>& queries) {
        vector<int> res;
        int even_sum = 0;
        for(int c:A)
        {
            if(c % 2 == 0) even_sum += c;
        } 
        for(auto i:queries)
        {
            if(A[i[1]] % 2 != 0 && i[0] % 2 == 0)  // odd && even
            {
                A[i[1]] += i[0];
                res.push_back(even_sum);
            }
            else if(A[i[1]] % 2 == 0 && i[0] % 2 != 0) // even && odd
            {
                even_sum -= A[i[1]];
                A[i[1]] += i[0];
                res.push_back(even_sum);
            }
            else if(A[i[1]] % 2 == 0 && i[0] % 2 == 0)    // even && even
            {
                even_sum += i[0];
                A[i[1]] += i[0];
                res.push_back(even_sum);
            }
            else                                         // odd && odd
            {
                A[i[1]] += i[0];
                even_sum += A[i[1]];
                res.push_back(even_sum);
                
            }
        }
        return res;
    }
};