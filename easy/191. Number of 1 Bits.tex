编写一个函数,输入是一个无符号整数,返回其二进制表达式中数字位数为 ‘1’ 的个数(也被称为汉明重量)。

 

示例 1:

输入:00000000000000000000000000001011
输出:3
解释:输入的二进制串 00000000000000000000000000001011 中,共有三位为 '1'。
示例 2:

输入:00000000000000000000000010000000
输出:1
解释:输入的二进制串 00000000000000000000000010000000 中,共有一位为 '1'。
示例 3:

输入:11111111111111111111111111111101
输出:31
解释:输入的二进制串 11111111111111111111111111111101 中,共有 31 位为 '1'。
 

提示:

请注意,在某些语言(如 Java)中,没有无符号整数类型。在这种情况下,输入和输出都将被指定为有符号整数类型,并且不应影响您的实现,因为无论整数是有符号的还是无符号的,其内部的二进制表示形式都是相同的。
在 Java 中,编译器使用二进制补码记法来表示有符号整数。因此,在上面的 示例 3 中,输入表示有符号整数 -3。






















思路:
if(n & 1) = 1, 则最后一位为 1,右移一位。




















code:
class Solution {
public:
    int hammingWeight(uint32_t n) {
        int ans = 0;
        int i = 32;
        while(i--)
        {
            if(n&1)
                ans++;
            n >>= 1;
        }
        return ans;
    }
};