给定一个初始元素全部为 0,大小为 m*n 的矩阵 M 以及在 M 上的一系列更新操作。

操作用二维数组表示,其中的每个操作用一个含有两个正整数 a 和 b 的数组表示,含义是将所有符合 0 <= i < a 以及 0 <= j < b 的元素 M[i][j] 的值都增加 1。

在执行给定的一系列操作后,你需要返回矩阵中含有最大整数的元素个数。

示例 1:

输入: 
m = 3, n = 3
operations = [[2,2],[3,3]]
输出: 4
解释: 
初始状态, M = 
[[0, 0, 0],
 [0, 0, 0],
 [0, 0, 0]]

执行完操作 [2,2] 后, M = 
[[1, 1, 0],
 [1, 1, 0],
 [0, 0, 0]]

执行完操作 [3,3] 后, M = 
[[2, 2, 1],
 [2, 2, 1],
 [1, 1, 1]]

M 中最大的整数是 2, 而且 M 中有4个值为2的元素。因此返回 4。

注意:

    m 和 n 的范围是 [1,40000]。
    a 的范围是 [1,m],b 的范围是 [1,n]。
    操作数目不超过 10000。




















思路:
连续操作的最大区域即为所求。





















code:
class Solution {
public:
    int maxCount(int m, int n, vector<vector<int>>& ops) {
        if(ops.size() == 0) return m*n;
        int maxr = 99999;
        int maxc = 99999;
        for(int i = 0; i < ops.size(); i++)
        {
            if(ops[i][0] < maxr) maxr = ops[i][0];
            if(ops[i][1] < maxc) maxc = ops[i][1];
        }
        return maxr*maxc;
    }
};