假设你正在爬楼梯。需要 n 阶你才能到达楼顶。

每次你可以爬 1 或 2 个台阶。你有多少种不同的方法可以爬到楼顶呢?

注意:给定 n 是一个正整数。

示例 1:

输入: 2
输出: 2
解释: 有两种方法可以爬到楼顶。
1.  1 阶 + 1 阶
2.  2 阶

示例 2:

输入: 3
输出: 3
解释: 有三种方法可以爬到楼顶。
1.  1 阶 + 1 阶 + 1 阶
2.  1 阶 + 2 阶
3.  2 阶 + 1 阶



























思路:
爬第n阶楼梯的方法数量,等于 2 部分之和

1.爬上 n−1 阶楼梯的方法数量。因为再爬1阶就能到第n阶
2.爬上 n−2 阶楼梯的方法数量,因为再爬2阶就能到第n阶

所以我们得到公式 dp[n]=dp[n−1]+dp[n−2]
同时需要初始化 dp[0]=1,dp[1]=2。






























code:
class Solution {
public:
    int climbStairs(int n) {
        if(n == 1) return 1;
        vector<int> res(n,0);
        res[0] = 1;
        res[1] = 2;
        for(int i = 2; i < n; i++)
            res[i] = res[i-1] + res[i-2];
        return res[n-1];
    }
};