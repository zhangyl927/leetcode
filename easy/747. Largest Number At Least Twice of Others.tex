在一个给定的数组nums中,总是存在一个最大元素 。

查找数组中的最大元素是否至少是数组中每个其他数字的两倍。

如果是,则返回最大元素的索引,否则返回-1。

示例 1:

输入: nums = [3, 6, 1, 0]
输出: 1
解释: 6是最大的整数, 对于数组中的其他整数,
6大于数组中其他元素的两倍。6的索引是1, 所以我们返回1.
 

示例 2:

输入: nums = [1, 2, 3, 4]
输出: -1
解释: 4没有超过3的两倍大, 所以我们返回 -1.
 

提示:

nums 的长度范围在[1, 50].
每个 nums[i] 的整数范围在 [0, 99].



























思路:
排序,最大的数 < 第二大的数 * 2, return -1; 否则遍历找到索引。



























code:
class Solution {
public:
    int dominantIndex(vector<int>& nums) {
        if(nums.size() == 1) return 0;
        vector<int> res = nums;
        sort(nums.begin(), nums.end());
        if(nums[nums.size()-1] < nums[nums.size()-2]*2)
            return -1;
        else
        {
            for(int i = 0; i < nums.size(); i++)
            {
                if(res[i] == nums[nums.size()-1])
                    return i;
            }
            return -1;
        }
    }
};