罗马数字包含以下七种字符: I, V, X, L,C,D 和 M。

字符          数值
I             1
V             5
X             10
L             50
C             100
D             500
M             1000

例如, 罗马数字 2 写做 II ,即为两个并列的 1。12 写做 XII ,即为 X + II 。 27 写做  XXVII, 即为 XX + V + II 。

通常情况下,罗马数字中小的数字在大的数字的右边。但也存在特例,例如 4 不写做 IIII,而是 IV。数字 1 在数字 5 的左边,所表示的数等于大数 5 减小数 1 得到的数值 4 。同样地,数字 9 表示为 IX。这个特殊的规则只适用于以下六种情况:

    I 可以放在 V (5) 和 X (10) 的左边,来表示 4 和 9。
    X 可以放在 L (50) 和 C (100) 的左边,来表示 40 和 90。 
    C 可以放在 D (500) 和 M (1000) 的左边,来表示 400 和 900。

给定一个罗马数字,将其转换成整数。输入确保在 1 到 3999 的范围内。

示例 1:

输入: "III"
输出: 3

示例 2:

输入: "IV"
输出: 4

示例 3:

输入: "IX"
输出: 9

示例 4:

输入: "LVIII"
输出: 58
解释: L = 50, V= 5, III = 3.

示例 5:

输入: "MCMXCIV"
输出: 1994
解释: M = 1000, CM = 900, XC = 90, IV = 4.















思路:
先将最后一位数拿出来为sum,再从倒数第二位 从后往前 遍历,若前一位 < 后一位, 如 IV,则 sum - I;若前一位 >= 后一位, 如 VI, 则 sum + V。
















code:
class Solution {
public:
    int romanToInt(string s) {
        if(s.size() == 0) return 0;
        unordered_map<char, int> T = {{'I', 1},
                                        {'V', 5},
                                        {'X', 10},
                                        {'L', 50},
                                        {'C', 100},
                                        {'D', 500},
                                        {'M', 1000}};
        int sum = T[s.back()];
        for(int i = s.size()-2; i >= 0; i--)
        {
            if(T[s[i]] < T[s[i+1]]) sum -= T[s[i]];
            else sum += T[s[i]];
        }
        return sum;
    }
};