给定一个包含大写字母和小写字母的字符串,找到通过这些字母构造成的最长的回文串。

在构造过程中,请注意区分大小写。比如 "Aa" 不能当做一个回文字符串。

注意:
假设字符串的长度不会超过 1010。

示例 1:

输入:
"abccccdd"

输出:
7

解释:
我们可以构造的最长的回文串是"dccaccd", 它的长度是 7。



























思路:
创建一个 map<char, int>,
1.当字符是偶数个时,可以全部用来组成 回文串。
2.当字符是奇数个时,减去 1 之后变成的偶数 可以全部用来组成回文串。
3.若 map 中没有奇数个数的,则将 结果返回,若有奇数个, 则将结果 + 1 返回。

























code:
class Solution {
public:
    int longestPalindrome(string s) {
        unordered_map<char, int> charmap;
        for(auto i:s)
            charmap[i]++;
        int length = 0, max_odd = 0, flag = 0;
        for(auto c:charmap)
        {
            if(c.second % 2 == 0)
                length += c.second;
            else
            {
                if(c.second > max_odd) length += c.second-1;
                flag = 1;
            }
        }
        return length+((flag == 1) ? 1 : 0);
    }
};