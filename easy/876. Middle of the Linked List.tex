给定一个带有头结点 head 的非空单链表,返回链表的中间结点。

如果有两个中间结点,则返回第二个中间结点。

 

示例 1:

输入:[1,2,3,4,5]
输出:此列表中的结点 3 (序列化形式:[3,4,5])
返回的结点值为 3 。 (测评系统对该结点序列化表述是 [3,4,5])。
注意,我们返回了一个 ListNode 类型的对象 ans,这样:
ans.val = 3, ans.next.val = 4, ans.next.next.val = 5, 以及 ans.next.next.next = NULL.
示例 2:

输入:[1,2,3,4,5,6]
输出:此列表中的结点 4 (序列化形式:[4,5,6])
由于该列表有两个中间结点,值分别为 3 和 4,我们返回第二个结点。
 

提示:

给定链表的结点数介于 1 和 100 之间。



























思路:
使用双指针,slow 指针每次移动一个位置, fast 指针每次移动两个位置;当 fast 指针移动到末尾时,slow 指针刚好指向中间元素的位置。



























code:
/**
 * Definition for singly-linked list.
 * struct ListNode {
 *     int val;
 *     ListNode *next;
 *     ListNode(int x) : val(x), next(NULL) {}
 * };
 */
class Solution {
public:
    ListNode* middleNode(ListNode* head) {
        ListNode* fast = head;
        ListNode* slow = head;
        while(fast)
        {
            if(!fast || !fast->next) return slow;
            slow = slow->next;
            fast = fast->next->next;
        }
        return slow;
    }
};