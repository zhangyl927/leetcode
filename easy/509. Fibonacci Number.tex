斐波那契数,通常用 F(n) 表示,形成的序列称为斐波那契数列。该数列由 0 和 1 开始,后面的每一项数字都是前面两项数字的和。也就是:

F(0) = 0,   F(1) = 1
F(N) = F(N - 1) + F(N - 2), 其中 N > 1.

给定 N,计算 F(N)。

 

示例 1:

输入:2
输出:1
解释:F(2) = F(1) + F(0) = 1 + 0 = 1.

示例 2:

输入:3
输出:2
解释:F(3) = F(2) + F(1) = 1 + 1 = 2.

示例 3:

输入:4
输出:3
解释:F(4) = F(3) + F(2) = 2 + 1 = 3.

 

提示:

    0 ≤ N ≤ 30








































思路:
递归。




































code:
class Solution {
public:
    int fib(int N) {
        if(N == 0 || N == 1) return N;
        return fib(N-1)+fib(N-2);
    }
};