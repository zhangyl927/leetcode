给定一个正整数,输出它的补数。补数是对该数的二进制表示取反。

注意:

给定的整数保证在32位带符号整数的范围内。
你可以假定二进制数不包含前导零位。
示例 1:

输入: 5
输出: 2
解释: 5的二进制表示为101(没有前导零位),其补数为010。所以你需要输出2。
示例 2:

输入: 1
输出: 0
解释: 1的二进制表示为1(没有前导零位),其补数为0。所以你需要输出0。



























思路:
各位取反放在容器中,计算结果。



























code:
class Solution {
public:
    int findComplement(int num) {
        int result = 0;
        vector<int> res;
        while(num)
        {
            if(num % 2 == 0) res.push_back(1);
            else res.push_back(0);
            num >>= 1;
        }
        
        for(int j = 0; j < res.size(); j++)
        {
            result += res[j]*pow(2,j);
        }
        return result;
    }
};