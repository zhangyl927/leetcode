给定一个二叉搜索树, 找到该树中两个指定节点的最近公共祖先。

百度百科中最近公共祖先的定义为:“对于有根树 T 的两个结点 p、q,最近公共祖先表示为一个结点 x,满足 x 是 p、q 的祖先且 x 的深度尽可能大(一个节点也可以是它自己的祖先)。”

例如,给定如下二叉搜索树:  root = [6,2,8,0,4,7,9,null,null,3,5]



 

示例 1:

输入: root = [6,2,8,0,4,7,9,null,null,3,5], p = 2, q = 8
输出: 6 
解释: 节点 2 和节点 8 的最近公共祖先是 6。
示例 2:

输入: root = [6,2,8,0,4,7,9,null,null,3,5], p = 2, q = 4
输出: 2
解释: 节点 2 和节点 4 的最近公共祖先是 2, 因为根据定义最近公共祖先节点可以为节点本身。

















思路:
若 root 为 p 或者 q,则结果为 root。
若 p->val < root->val < q->val 或者 q->val < root->val < p->val,  由二叉搜索树的属性,结果为 root;
若 p->val < root->val &&  q->val < root->val, root = root->left;
若 p->val > root->val &&  q->val > root->val, root = root->right;




























code:
/**
 * Definition for a binary tree node.
 * struct TreeNode {
 *     int val;
 *     TreeNode *left;
 *     TreeNode *right;
 *     TreeNode(int x) : val(x), left(NULL), right(NULL) {}
 * };
 */
class Solution {
public:
    TreeNode* lowestCommonAncestor(TreeNode* root, TreeNode* p, TreeNode* q) {
        if(root->val == p->val || root->val == q->val) return root;
        else if((p->val < root->val && root->val < q->val) || (p->val > root->val && root->val > q->val))
            return root;
        else if(p->val < root->val && root->val > q->val)
            return lowestCommonAncestor(root->left, p, q);
        else return lowestCommonAncestor(root->right, p, q);
    }
};