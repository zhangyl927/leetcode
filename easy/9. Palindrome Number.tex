判断一个整数是否是回文数。回文数是指正序(从左向右)和倒序(从右向左)读都是一样的整数。

示例 1:

输入: 121
输出: true
示例 2:

输入: -121
输出: false
解释: 从左向右读, 为 -121 。 从右向左读, 为 121- 。因此它不是一个回文数。
示例 3:

输入: 10
输出: false
解释: 从右向左读, 为 01 。因此它不是一个回文数。


思路:
转换为字符串,翻转字符串与原来的数比较。















code:
class Solution {
public:
    bool isPalindrome(int x) {
        if(x < 0 || x % 10 == 0 && x != 0) return false;
        string s = to_string(x);
        string r_s = s;
        std::reverse(s.begin(), s.end());
        if(r_s != s) return false;
        return true;
    }
};