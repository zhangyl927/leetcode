给定一个整数 n,返回 n! 结果尾数中零的数量。

示例 1:

输入: 3
输出: 0
解释: 3! = 6, 尾数中没有零。
示例 2:

输入: 5
输出: 1
解释: 5! = 120, 尾数中有 1 个零.
说明: 你算法的时间复杂度应为 O(log n) 。



















思路:
由观察可知,n < 5 时,阶乘没有0,5 <= n < 10 时,有一个0...... 归纳总结,0的个数与 5 的数目相关。则求出 阶乘中 可以拆分为多少个5相乘即可。即 n/5。
注意 当 n 为 5 的次方时,有多个5存在。












code:
class Solution {
public:
    int trailingZeroes(int n) {
        int number = 0;
        while(n >= 1)
        {
            number += n / 5;
            n = n / 5;
        }
        return number;
    }
};