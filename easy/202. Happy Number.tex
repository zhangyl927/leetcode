编写一个算法来判断一个数是不是“快乐数”。

一个“快乐数”定义为:对于一个正整数,每一次将该数替换为它每个位置上的数字的平方和,然后重复这个过程直到这个数变为 1,也可能是无限循环但始终变不到 1。如果可以变为 1,那么这个数就是快乐数。

示例: 

输入: 19
输出: true
解释: 
12 + 92 = 82
82 + 22 = 68
62 + 82 = 100
12 + 02 + 02 = 1

























思路:
若递归计算20次还不符合,则认为该数不是快乐数。
























code:
class Solution {
public:
    bool isHappy(int n) {
        return judgeHappy(n, 0); 
    }
    bool judgeHappy(int n, int depth)
    {
        if(n == 1) return true;
        if(depth > 20) return false;
        int temp = 0;
        while(n > 0)
        {
            temp += pow(n%10,2);
            n /= 10;
        }
        depth++;
        return judgeHappy(temp, depth); 
    }
};