给定一个二叉树,返回所有从根节点到叶子节点的路径。

说明: 叶子节点是指没有子节点的节点。

示例:

输入:

   1
 /   \
2     3
 \
  5

输出: ["1->2->5", "1->3"]

解释: 所有根节点到叶子节点的路径为: 1->2->5, 1->3





























思路:
遍历每一个节点,当到达根时,将路径保存,回溯找另外的路径。




































code:
/**
 * Definition for a binary tree node.
 * struct TreeNode {
 *     int val;
 *     TreeNode *left;
 *     TreeNode *right;
 *     TreeNode(int x) : val(x), left(NULL), right(NULL) {}
 * };
 */
class Solution {
public:
    vector<string> binaryTreePaths(TreeNode* root) {
        vector<string> res;
        string str = "";
        searchPath(root, str, res);
        return res;
    }
    void searchPath(TreeNode* Node, string s, vector<string>& res) // 由于是递归,使用 s 不需要引用
    {
        if(!Node) return;
        if(!Node->left && !Node->right)
        {
            s += to_string(Node->val);
            res.push_back(s);
        }
        else s += to_string(Node->val);

        searchPath(Node->left, s+"->", res);
        searchPath(Node->right, s+"->", res);
        s.pop_back();
    }
};