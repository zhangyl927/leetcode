给定一个正整数 N,找到并返回 N 的二进制表示中两个连续的 1 之间的最长距离。 

如果没有两个连续的 1,返回 0 。

 

示例 1:

输入:22
输出:2
解释:
22 的二进制是 0b10110 。
在 22 的二进制表示中,有三个 1,组成两对连续的 1 。
第一对连续的 1 中,两个 1 之间的距离为 2 。
第二对连续的 1 中,两个 1 之间的距离为 1 。
答案取两个距离之中最大的,也就是 2 。

示例 2:

输入:5
输出:2
解释:
5 的二进制是 0b101 。

示例 3:

输入:6
输出:1
解释:
6 的二进制是 0b110 。

示例 4:

输入:8
输出:0
解释:
8 的二进制是 0b1000 。
在 8 的二进制表示中没有连续的 1,所以返回 0 。

 

提示:

    1 <= N <= 10^9

























思路:
将整数转换为 二进制 保存在 容器中,遍历容器,找到两个 1 之间的最大值即可。



























code:
class Solution {
public:
    int binaryGap(int N) {
        vector<int> binary;
        while(N)
        {
            if(N % 2 == 0) binary.push_back(0);
            else binary.push_back(1);
            N >>= 1;
        }
        
        int left = 0, right = 0, maxDistance = 0, flag = 0;
        for(int i = 0; i < binary.size(); i++)
        {
            if(binary[i] == 1 && flag == 0)
            {
                flag = 1; left = i; right = i;
            }
            if(binary[i] == 1 && flag == 1)
            {
                left = right; right = i;
                if((right-left) > maxDistance) maxDistance = right-left;
            }
        }
        return maxDistance;
    }
};