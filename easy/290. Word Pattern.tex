给定一种规律 pattern 和一个字符串 str ,判断 str 是否遵循相同的规律。

这里的 遵循 指完全匹配,例如, pattern 里的每个字母和字符串 str 中的每个非空单词之间存在着双向连接的对应规律。

示例1:

输入: pattern = "abba", str = "dog cat cat dog"
输出: true

示例 2:

输入:pattern = "abba", str = "dog cat cat fish"
输出: false

示例 3:

输入: pattern = "aaaa", str = "dog cat cat dog"
输出: false

示例 4:

输入: pattern = "abba", str = "dog dog dog dog"
输出: false

说明:
你可以假设 pattern 只包含小写字母, str 包含了由单个空格分隔的小写字母。    

























思路:
1.istringstream in(str) 实现 java 的 字符串切割 split。使用 hash。
2.istringstream in(str) 实现字符串切割,若 pattern.length() != sv.size(),则 false;;// 如 juquery, juquery
遍历 pattern,找当前元素出现的第一个 position,若 pattern[i]的 position != sv[i] 的 position,返回false。




















code2:
class Solution {
public:
    bool wordPattern(string pattern, string str) {
        istringstream in(str);
        vector<string> sv;
        string temp;
        while(in >> temp) sv.push_back(temp);
        
        if(pattern.length() != sv.size()) return false;
        
        for(int i = 0; i < pattern.size(); i++)
        {
            int ppos = pattern.find(pattern[i]);
            int spos = find(sv.begin(), sv.end(), sv[i])-sv.begin();
            if(ppos != spos) return false;
        }
        return true;
    }
};













code:
class Solution {
public:
    bool wordPattern(string pattern, string str) {
        istringstream in(str);
        vector<string> v;
        string t;
        while(in >> t) v.push_back(t);

        unordered_map<char, string> map1;
        if(pattern.size() != v.size()) return false;
        for(int i = 0; i < pattern.size(); i++)
        {
            char c = pattern[i];
            if(map1.count(c))
            {
                if(map1[c] != v[i]) return false;
            }
            else
            {
                for(auto j:map1)
                {
                    if(j.second == v[i]) return false;
                }
                map1[c] = v[i];
            }
        }
        return true;
    }
};