给定一个非负整数 num,反复将各个位上的数字相加,直到结果为一位数。

示例:

输入: 38
输出: 2 
解释: 各位相加的过程为:3 + 8 = 11, 1 + 1 = 2。 由于 2 是一位数,所以返回 2。(代码可以优化)


















思路:
设计计算 num 位数的函数,当位数小于 9 时即为结果,否则递归。























优化code:
class Solution {
public:
    int addDigits(int num) {
        if(num < 10) return num;
        int result = 0;
        while(num)
        {
            result += num % 10;
            num /= 10;
        }
        return addDigits(result);
    }
};

















原code:
class Solution {
public:
    int addDigits(int num) {
        int time = getCount(num);
        return getResult(num, time);
    }
    int getCount(int num)
    {
        int count = 0;
        while(num)
        {
            num /= 10;
            count++;
        }
        return count;
    }
    int getResult(int num, int time)
    {
        int Result = 0;
        while(time--)
        {
            Result += num % 10;
            num /= 10;
        }
        if(Result > 9)
            return getResult(Result, getCount(Result));
        else return Result;
    }
};