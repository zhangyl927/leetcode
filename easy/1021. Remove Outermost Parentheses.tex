有效括号字符串为空 ("")、"(" + A + ")" 或 A + B,其中 A 和 B 都是有效的括号字符串,+ 代表字符串的连接。例如,"","()","(())()" 和 "(()(()))" 都是有效的括号字符串。

如果有效字符串 S 非空,且不存在将其拆分为 S = A+B 的方法,我们称其为原语(primitive),其中 A 和 B 都是非空有效括号字符串。

给出一个非空有效字符串 S,考虑将其进行原语化分解,使得:S = P_1 + P_2 + ... + P_k,其中 P_i 是有效括号字符串原语。

对 S 进行原语化分解,删除分解中每个原语字符串的最外层括号,返回 S 。

 

示例 1:

输入:"(()())(())"
输出:"()()()"
解释:
输入字符串为 "(()())(())",原语化分解得到 "(()())" + "(())",
删除每个部分中的最外层括号后得到 "()()" + "()" = "()()()"。

示例 2:

输入:"(()())(())(()(()))"
输出:"()()()()(())"
解释:
输入字符串为 "(()())(())(()(()))",原语化分解得到 "(()())" + "(())" + "(()(()))",
删除每隔部分中的最外层括号后得到 "()()" + "()" + "()(())" = "()()()()(())"。

示例 3:

输入:"()()"
输出:""
解释:
输入字符串为 "()()",原语化分解得到 "()" + "()",
删除每个部分中的最外层括号后得到 "" + "" = ""。

 

提示:

    S.length <= 10000
    S[i] 为 "(" 或 ")"
    S 是一个有效括号字符串


























思路:
从左往右遍历,当左边遇到第 2 个 '(' 时,即将其输出;
可以使用计数的方法,没遇到一个 ’(‘ 时,count + 1,每遇到一个 ’)‘时,count - 1;




























code:
class Solution {
public:
    string removeOuterParentheses(string S) {
        int count = 0;
        string res = "";
        for(int i = 0; i < S.size(); i++)
        {
            if(S[i] == '(')
            {
                count++;
                if(count > 1) res += S[i];
            }
            else
            {
                count--;
                if(count >= 1) res += S[i];
            }
        }
        return res;
    }
};