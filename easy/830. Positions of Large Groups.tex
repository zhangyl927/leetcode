在一个由小写字母构成的字符串 S 中,包含由一些连续的相同字符所构成的分组。

例如,在字符串 S = "abbxxxxzyy" 中,就含有 "a", "bb", "xxxx", "z" 和 "yy" 这样的一些分组。

我们称所有包含大于或等于三个连续字符的分组为较大分组。找到每一个较大分组的起始和终止位置。

最终结果按照字典顺序输出。

示例 1:

输入: "abbxxxxzzy"
输出: [[3,6]]
解释: "xxxx" 是一个起始于 3 且终止于 6 的较大分组。

示例 2:

输入: "abc"
输出: []
解释: "a","b" 和 "c" 均不是符合要求的较大分组。

示例 3:

输入: "abcdddeeeeaabbbcd"
输出: [[3,5],[6,9],[12,14]]

说明:  1 <= S.length <= 1000

























思路:
暴力法。






















code:
class Solution {
public:
    vector<vector<int>> largeGroupPositions(string S) {
        vector<vector<int>> res;
        vector<int> result;
        for(int i = 0; i < S.length(); i++)
        {
            int temp = 1;
            while(i+1 < S.length() && S[i] == S[i+1])
            {
                i++;
                temp++;
            }
            if(temp >= 3)
            {
                result.push_back(i-temp+1);
                result.push_back(i);
                res.push_back(result);
                result.pop_back();
                result.pop_back();
            }
            if(i >= S.length()-1) break;
        }
        return res;
    }
};