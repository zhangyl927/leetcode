爱丽丝和鲍勃有不同大小的糖果棒:A[i] 是爱丽丝拥有的第 i 块糖的大小,B[j] 是鲍勃拥有的第 j 块糖的大小。

因为他们是朋友,所以他们想交换一个糖果棒,这样交换后,他们都有相同的糖果总量。(一个人拥有的糖果总量是他们拥有的糖果棒大小的总和。)

返回一个整数数组 ans,其中 ans[0] 是爱丽丝必须交换的糖果棒的大小,ans[1] 是 Bob 必须交换的糖果棒的大小。

如果有多个答案,你可以返回其中任何一个。保证答案存在。

 

示例 1:

输入:A = [1,1], B = [2,2]
输出:[1,2]

示例 2:

输入:A = [1,2], B = [2,3]
输出:[1,2]

示例 3:

输入:A = [2], B = [1,3]
输出:[2,3]

示例 4:

输入:A = [1,2,5], B = [2,4]
输出:[5,4]

 

提示:

    1 <= A.length <= 10000
    1 <= B.length <= 10000
    1 <= A[i] <= 100000
    1 <= B[i] <= 100000
    保证爱丽丝与鲍勃的糖果总量不同。
    答案肯定存在。























思路:
在两个数组中,找相差(A_sum - B_sum) / 2 的两个数即为答案。


















code:
class Solution {
public:
    vector<int> fairCandySwap(vector<int>& A, vector<int>& B) {
        int diff = (accumulate(A.begin(), A.end(), 0) - accumulate(B.begin(), B.end(), 0)) / 2;
        unordered_set<int> res(A.begin(), A.end());
        for(int c:B)
        {
            if(res.count(c + diff))
                return {c+diff, c};
        }
        return {};
    }
};