给定仅有小写字母组成的字符串数组 A,返回列表中的每个字符串中都显示的全部字符(包括重复字符)组成的列表。例如,如果一个字符在每个字符串中出现 3 次,但不是 4 次,则需要在最终答案中包含该字符 3 次。

你可以按任意顺序返回答案。

 

示例 1:

输入:["bella","label","roller"]
输出:["e","l","l"]

示例 2:

输入:["cool","lock","cook"]
输出:["c","o"]

 

提示:

    1 <= A.length <= 100
    1 <= A[i].length <= 100
    A[i][j] 是小写字母

































思路:
将 A[0] 的元素放入 容器 c 中,依次将 A[1] - A[A.size()-1] 中的元素放入容器后,与 c 中元素的比较,将两者的交 结果更新 c。 


































code:
class Solution {
public:
    vector<string> commonChars(vector<string>& A) {
        vector<string> res;
        vector<int> c(26,0);
        for(auto a:A[0])
            ++c[a-'a'];
        for(int i = 1; i < A.size(); i++)
        {
            vector<int> temp(26, 0);
            for(int j = 0; j < A[i].size(); j++)
            {
                ++temp[A[i][j] - 'a'];
            }
            for(int k = 0; k < 26; k++)
            {
                c[k] = (c[k] > temp[k]) ? temp[k] : c[k];
            }
        }
        for(int m = 0; m < 26; m++)
        {
            while(c[m] > 0)
            {
                res.push_back(string(1, m + 'a'));
                c[m]--;
            }
        }
        return res;
    }
};