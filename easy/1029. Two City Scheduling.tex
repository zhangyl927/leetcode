公司计划面试 2N 人。第 i 人飞往 A 市的费用为 costs[i][0],飞往 B 市的费用为 costs[i][1]。

返回将每个人都飞到某座城市的最低费用,要求每个城市都有 N 人抵达。

 

示例:

输入:[[10,20],[30,200],[400,50],[30,20]]
输出:110
解释:
第一个人去 A 市,费用为 10。
第二个人去 A 市,费用为 30。
第三个人去 B 市,费用为 50。
第四个人去 B 市,费用为 20。

最低总费用为 10 + 30 + 50 + 20 = 110,每个城市都有一半的人在面试。

 

提示:

    1 <= costs.length <= 100
    costs.length 为偶数
    1 <= costs[i][0], costs[i][1] <= 1000


























思路:
lambda。
https://www.youtube.com/watch?v=482weZjwVHY





























code:
class Solution {
public:
    int twoCitySchedCost(vector<vector<int>>& costs) {
        int sum = 0;
        sort(costs.begin(), costs.end(), [](vector<int>& v1, vector<int>& v2)
        {
            return (v1[0]-v1[1] < v2[0]-v2[1]);
        });
        for(int i = 0; i < costs.size()/2; i++)
        {
            sum += costs[i][0] + costs[i+costs.size()/2][1];
        }
        return sum;
    }
};