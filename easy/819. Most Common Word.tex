给定一个段落 (paragraph) 和一个禁用单词列表 (banned)。返回出现次数最多,同时不在禁用列表中的单词。题目保证至少有一个词不在禁用列表中,而且答案唯一。

禁用列表中的单词用小写字母表示,不含标点符号。段落中的单词不区分大小写。答案都是小写字母。

 

示例:

输入: 
paragraph = "Bob hit a ball, the hit BALL flew far after it was hit."
banned = ["hit"]
输出: "ball"
解释: 
"hit" 出现了3次,但它是一个禁用的单词。
"ball" 出现了2次 (同时没有其他单词出现2次),所以它是段落里出现次数最多的,且不在禁用列表中的单词。 
注意,所有这些单词在段落里不区分大小写,标点符号需要忽略(即使是紧挨着单词也忽略, 比如 "ball,"), 
"hit"不是最终的答案,虽然它出现次数更多,但它在禁用单词列表中。

 

说明:

    1 <= 段落长度 <= 1000.
    1 <= 禁用单词个数 <= 100.
    1 <= 禁用单词长度 <= 10.
    答案是唯一的, 且都是小写字母 (即使在 paragraph 里是大写的,即使是一些特定的名词,答案都是小写的。)
    paragraph 只包含字母、空格和下列标点符号!?',;.
    不存在没有连字符或者带有连字符的单词。
    单词里只包含字母,不会出现省略号或者其他标点符号。
































思路:
1.全部转换为小写字母,并将其中的非字母替换为 ‘ ’。
2.将字符串存入容器中,并在 map 中记录每个单词出现的次数。
3.遍历 banned 中的单词,若单词在 map 中出现,则将 map 中该单词的次数清零。
4.遍历 map,查找 map 中次数最多的那个单词。






























code:
class Solution {
public:
    string mostCommonWord(string paragraph, vector<string>& banned) {
        unordered_map<string, int> myMap;
        //  全部变为小写
        for(auto &c:paragraph)
        {
            if(c >= 'A' && c <= 'Z') c += 32;
            if(!isalpha(c)) c = ' ';
        }

        //  统计各单词出现的次数
        istringstream in(paragraph);
        string t;
        vector<string> pv;
        while(in >> t)
        {
            pv.push_back(t);
        }
        
        for(auto i:pv)
            ++myMap[i];
        
        //   出现在 banned 中的单词次数都清零
        for(auto i:banned)
        {
            if(myMap.count(i)) myMap[i] = 0;
        }
        
        int max = 0;
        string res;
        for(auto k:myMap)
        {
            if(k.second > max) max = k.second;
        }
        for(auto m:myMap)
        {
            if(m.second == max) res = m.first;
        }
        return res;
    }
};