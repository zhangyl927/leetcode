翻转一棵二叉树。

示例:

输入:

     4
   /   \
  2     7
 / \   / \
1   3 6   9
输出:

     4
   /   \
  7     2
 / \   / \
9   6 3   1

















思路:
1. 交换节点的左子树和右子树。
2. BFS。


















code1-1:
/**
 * Definition for a binary tree node.
 * struct TreeNode {
 *     int val;
 *     TreeNode *left;
 *     TreeNode *right;
 *     TreeNode(int x) : val(x), left(NULL), right(NULL) {}
 * };
 */
class Solution {
public:
    TreeNode* invertTree(TreeNode* root) {
        if(!root) return NULL;
        TreeNode* temp = root->left;
        root->left = root->right;
        root->right = temp;
        
        if(root->left || root->right)
        {
            invertTree(root->left);
            invertTree(root->right);
        }
        return root;
    }
};



















code1-2:
/**
 * Definition for a binary tree node.
 * struct TreeNode {
 *     int val;
 *     TreeNode *left;
 *     TreeNode *right;
 *     TreeNode(int x) : val(x), left(NULL), right(NULL) {}
 * };
 */
class Solution {
public:
    TreeNode* invertTree(TreeNode* root) {
        invert(root);
        return root;
    }
    void invert(TreeNode* Node)
    {
        if(!Node) return;
        swap(Node->left, Node->right);
        invert(Node->left);
        invert(Node->right);
    }
};























code2:
/**
 * Definition for a binary tree node.
 * struct TreeNode {
 *     int val;
 *     TreeNode *left;
 *     TreeNode *right;
 *     TreeNode(int x) : val(x), left(NULL), right(NULL) {}
 * };
 */
class Solution {
public:
    TreeNode* invertTree(TreeNode* root) {
        if(!root) return root;
        queue<TreeNode*> myQueue;
        myQueue.push(root);
        while(!myQueue.empty())
        {
            TreeNode* node = myQueue.front();
            TreeNode* temp = node->left;
            node->left = node->right;
            node->right = temp;
            myQueue.pop();
            
            if(node->left) myQueue.push(node->left);
            if(node->right) myQueue.push(node->right);
        }
        return root;
    }
};