给定一个数组,它的第 i 个元素是一支给定股票第 i 天的价格。

设计一个算法来计算你所能获取的最大利润。你可以尽可能地完成更多的交易(多次买卖一支股票)。

注意:你不能同时参与多笔交易(你必须在再次购买前出售掉之前的股票)。

示例 1:

输入: [7,1,5,3,6,4]
输出: 7
解释: 在第 2 天(股票价格 = 1)的时候买入,在第 3 天(股票价格 = 5)的时候卖出, 这笔交易所能获得利润 = 5-1 = 4 。
     随后,在第 4 天(股票价格 = 3)的时候买入,在第 5 天(股票价格 = 6)的时候卖出, 这笔交易所能获得利润 = 6-3 = 3 。

示例 2:

输入: [1,2,3,4,5]
输出: 4
解释: 在第 1 天(股票价格 = 1)的时候买入,在第 5 天 (股票价格 = 5)的时候卖出, 这笔交易所能获得利润 = 5-1 = 4 。
     注意你不能在第 1 天和第 2 天接连购买股票,之后再将它们卖出。
     因为这样属于同时参与了多笔交易,你必须在再次购买前出售掉之前的股票。

示例 3:

输入: [7,6,4,3,1]
输出: 0
解释: 在这种情况下, 没有交易完成, 所以最大利润为 0。






























思路:
1.后一天的比前一天的大,则买入。
2.股票系列框架解法;















框架解法code:
class Solution {
public:
    int maxProfit(vector<int>& prices) {
        int dp_i_0 = 0, dp_i_1 = INT_MIN;
        for(int i = 0; i < prices.size(); i++)
        {
            dp_i_0 = max(dp_i_0, dp_i_1+prices[i]);
            dp_i_1 = max(dp_i_1, dp_i_0-prices[i]);
        }
        return dp_i_0;
    }
};






















code:
class Solution {
public:
    int maxProfit(vector<int>& prices) {
        int profit = 0;
        for(int i = 1; i < prices.size(); i++)
        {
            if(prices[i] - prices[i-1] > 0)
                profit += prices[i] - prices[i-1];
        }
        return profit;
    }
};