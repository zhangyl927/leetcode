对于一个 正整数,如果它和除了它自身以外的所有正因子之和相等,我们称它为“完美数”。

给定一个 正整数 n, 如果他是完美数,返回 True,否则返回 False

 

示例:

输入: 28
输出: True
解释: 28 = 1 + 2 + 4 + 7 + 14

 

注意:

输入的数字 n 不会超过 100,000,000. (1e8)

































思路:
遍历 0 - sqrt(num),找出因子,放入容器中。






























code:
class Solution {
public:
    bool checkPerfectNumber(int num) {
        vector<int> res;
        if(num == 1) return false;
        for(int i = 2; i <= sqrt(num); i++)
        {
            if(num % i == 0)
            {
                res.push_back(i);
                res.push_back(num/i);
            }
        }
        int sum = 0;
        for(auto j:res)
            sum += j;
        return num == sum+1;
    }
};