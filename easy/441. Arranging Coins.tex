你总共有 n 枚硬币,你需要将它们摆成一个阶梯形状,第 k 行就必须正好有 k 枚硬币。

给定一个数字 n,找出可形成完整阶梯行的总行数。

n 是一个非负整数,并且在32位有符号整型的范围内。

示例 1:

n = 5

硬币可排列成以下几行:
¤
¤ ¤
¤ ¤

因为第三行不完整,所以返回2.

示例 2:

n = 8

硬币可排列成以下几行:
¤
¤ ¤
¤ ¤ ¤
¤ ¤

因为第四行不完整,所以返回3.





















思路:
二分法。




















code:
class Solution {
public:
    int arrangeCoins(int n) {
        return search(n, 1, n);
    }
    int search(int n, long int begin, long int end)
    {
        if(begin == end && getsum(begin) <= n) return begin;
        if(begin == end && getsum(begin) > n) return begin-1;
        long int middle = begin + (end-begin)/2;
        if(getsum(middle) == n) return middle;
        else if(getsum(middle) > n) return search(n, begin, middle);
        else return search(n, middle+1, end);
    }
    long int getsum(long int x)
    {
        long int temp = 0;
        if(x % 2 == 0) temp += x + (x/2)*(x-1);
        else temp += x + ((x-1)/2)*x;
        return temp;
    }
};