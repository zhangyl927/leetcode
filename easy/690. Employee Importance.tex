给定一个保存员工信息的数据结构,它包含了员工唯一的id,重要度 和 直系下属的id。

比如,员工1是员工2的领导,员工2是员工3的领导。他们相应的重要度为15, 10, 5。那么员工1的数据结构是[1, 15, [2]],员工2的数据结构是[2, 10, [3]],员工3的数据结构是[3, 5, []]。注意虽然员工3也是员工1的一个下属,但是由于并不是直系下属,因此没有体现在员工1的数据结构中。

现在输入一个公司的所有员工信息,以及单个员工id,返回这个员工和他所有下属的重要度之和。

示例 1:

输入: [[1, 5, [2, 3]], [2, 3, []], [3, 3, []]], 1
输出: 11
解释:
员工1自身的重要度是5,他有两个直系下属2和3,而且2和3的重要度均为3。因此员工1的总重要度是 5 + 3 + 3 = 11。

注意:

    一个员工最多有一个直系领导,但是可以有多个直系下属
    员工数量不超过2000。




























思路:
创建一个 map,key = id, value = Employees*。
先根据 id 在map 中找到其对应的 Employee*,计算出该成员的重要性即可。


























code:
/*
// Employee info
class Employee {
public:
    // It's the unique ID of each node.
    // unique id of this employee
    int id;
    // the importance value of this employee
    int importance;
    // the id of direct subordinates
    vector<int> subordinates;
};
*/
class Solution {
public:
    int getImportance(vector<Employee*> employees, int id) {
        unordered_map<int, Employee*> e_map;
        for(auto employee:employees)
            e_map[employee->id] = employee;
        
        return getImportanceHelper(e_map, id);
    }
    int getImportanceHelper(unordered_map<int, Employee*> e_map, int id)
    {
        Employee* root = e_map.at(id);
        int total = root->importance;
        for(int i:root->subordinates)
            total += getImportanceHelper(e_map, i);
        return total;
    }
};